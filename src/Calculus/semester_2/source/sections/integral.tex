\subsection{Неопределенный интеграл.}

 Дано: $f: \langle a,b\rangle \rightarrow \mathbb{R}$. $F$ называется \deff{первообразной} функции $f$, если:

\begin{enumerate}
    \item $F$ дифференцируема на $\langle a,b\rangle$.

    \item $\forall x \in \langle a,b\rangle: F'(x) = f(x) $.
\end{enumerate}

\thmm{Теорема 1}

$f$ - непрерывна на $\langle a,b \rangle$. Тогда $f$ имеет первообразную на $\langle a,b\rangle$.

\textbf{Доказательство:}
\begin{quote}
    <см теорема Барроу>
    
    \hfill Q.E.D.
\end{quote}

\thmm{Теорема 2}

$F$ - первообразная $f$ на $\langle a,b \rangle$. Тогда:
\begin{enumerate}
    \item $\forall c \in \mathbb{R}$: $F + c$ тоже первообразная.
    \item Если $G$ - еще одна первообразная f, то $F - G = const$.
\end{enumerate}
\textbf{Доказательство:}

    \begin{enumerate}
        \item Воспользуемся арифметическим свойством производной. Тривиально.
        \item $(F-G)' = F' -G' =f-f= 0$. Пользуясь теоремами, так как производная везде $\geq 0$, то $F-G$ неубывающая. Аналогично так как производная на промежутке $\leq 0 $, то $F-G$ невозврастающая. Откуда это константа.
    \end{enumerate}
    \hfill Q.E.D.

\deff{Неопределенный интеграл} $f$ --- это множество всех первообразных $f$.

\textbf{Замечание от Славы.} Кохась подразумевает, что неопределенный интеграл это множество всех первообразных на том же интервале $\langle a,b \rangle$.

Обозначается неопределенный интеграл так:
$$\int f \quad \text{или} \int f(x) dx $$

Формально: $\displaystyle \int f(x)dx = F(x) +C$

\deff{Таблица неопределенных интегралов:}

Она переписывается из таблицы производных, просто в обратную сторону. Но есть две \uline{\emph{загадочные}} формулы:
     $$\displaystyle \int \cfrac{dx}{1-x^2} = \frac{1}{2}\ln \left|\cfrac{1+x}{1-x}\right| + C \quad \quad \int \cfrac{dx}{\sqrt{1+x^2}} = \ln \left|x + \sqrt{1+x^2}\right| + C$$

\thmm{Теорема (о св-вах неопределенного интервала)}

Пусть $f,g$ - имеют первообразные $F,G$ на $\langle a,b \rangle$. Тогда:

\begin{enumerate}
    \item $\dint (f+g) = \dint f + \dint g$
    \item $\forall a \in \mathbb{R}: \dint (af)=a\dint f$
    \item $\dint f(\varphi(t))\varphi'(t)dt = \left(\dint f(x)dx\right)\Big |_{x=\varphi(t)}= F(\varphi(t)) + C$
    \item частный случай. $\forall \alpha,\beta \in \mathbb{R}:\dint f(\alpha x + \beta) = \frac{1}{\alpha}F(\alpha x+\beta) + C$
    \item $f,g$ - дифф. на $\langle a,b\rangle$. Пусть $f'g$ и $fg'$ имеют первообразную: 
    
    Тогда: $\dint f g' = fg -\dint f'g$
\end{enumerate}

\textbf{Доказательство:}
\begin{enumerate}
    \item Очевидно из свойств производной и теоремы 2.
    \item Очевидно из свойств производной и теоремы 2.
    \item Очевидно из производной композиции.
    \item Очевидно из свойств производной и теоремы 2.
    \item Перенесите интеграл в правой части налево. Очевидно из произведения производных.
\end{enumerate}
    \hfill Q.E.D.

\textbf{Замечание.} Формула 3 часто будет использоваться для замены переменных в интегралах.
$$F(\varphi(t)) = \dint f(\varphi(t))\varphi'(t)dt $$
Давайте считать, что $\varphi $ обратима. Тогда $t = \varphi^{-1}(x)$. Подставим:
$$F(x)=\left(\dint f(\varphi(t))\varphi'(t)dt\right)\Big|_{t: = \varphi^{-1}(x)} $$

Для чего это? Благодаря этому, мы умеем вычислять первообразные немного по-другому. Мы можем подставлять вместо $x$ что-либо, а потом возвращаться обратно к $x$.

\subsection{Выпуклые функции.}

Множество $A \subset R^m$ \deff{выпукло}, если:
$$\forall x,y \in A , [x,y]\subset A: [x,y] = \{x+t(y-x), t\in[0,1]\} = \{(1-t)x + ty, t\in[0,1]\}$$
 $f:\langle a,b\rangle\rightarrow \mathbb{R}$ --- \deff{выпукла} на промежутке $\langle a,b\rangle$, если:
$$\forall x_1,x_2 \in \langle a,b \rangle: \forall \alpha \in[0,1]:f(\alpha x_1+(1-\alpha)x_2)\leq \alpha f(x_1) + (1-\alpha) f(x_2)$$

\deff{Надграфик} ($f$, $\langle c,d \rangle) = \{(x,y): x\in \langle c,d \rangle, y \geq f(x)  \}$

\textbf{Замечание.} $f$ - выпукло на $\langle a,b\rangle \Leftrightarrow$ Надграфик $(f, \langle a,b \rangle)$ - выпуклый в $R^2$.

\thmm{Лемма (о трех хордах)}

$f$ - выпукла на $\langle a,b\rangle \Leftrightarrow \forall x_1<x_2<x_3 \in\langle a,b \rangle$ выполнено:
$$\cfrac{f(x_2)-f(x_1)}{x_2-x_1} \leq \cfrac{f(x_3)-f(x_1)}{x_3-x_1}\leq \cfrac{f(x_3)-f(x_2)}{x_3-x_2}$$
\textbf{Доказательство:}

Возьму первое неравенство. Домножу на знаменатели и оставлю плюсы:
$$(x_3-x_1) (f(x_2)-f(x_1)) \leq (x_2-x_1) (f(x_3)-f(x_1))$$
$$f(x_2)\leq \cfrac{x_2-x_1}{x_3-x_1}f(x_3) + \cfrac{x_3-x_2}{x_3-x_1}f(x_1)$$
Чего-то не хватает, вспомним, что $f(x_2) = f\left (\cfrac{x_2-x_1}{x_3-x_1}x_3 +\cfrac{x_3-x_2}{x_3-x_1}x_1\right)$. Ой, это же условие выпуклости. Так как все переходы равносильны, то 
это неравенство выполнено, когда $f$ выпукла. Второе неравенство решается аналогично (позже будет добавлено в конспект).

\hfill Q.E.D.

$f$ - \deff{строго выпукла} на $\langle a,b\rangle$:
$$\forall x_1,x_2 \in \langle a,b \rangle: \forall \alpha \in[0,1]:f(\alpha x_1+(1-\alpha)x_2)< \alpha f(x_1) + (1-\alpha) f(x_2)$$
Просто меняется знак на строгий.

\thmm{Теорема (об одностор. дифф-ти вып. функции)}

$f$ - выпукла на $\langle a,b \rangle$. Тогда $\forall x\in( a,b): \exists f'_+(x), f'_-(x)$ (конечные),  а также

$\forall x_1<x_2\in \langle a,b\rangle$ выполнено:
$$f'_-(x_1)\leq f'_+(x_1) \leq \cfrac{f(x_2)-f(x_1)}{x_2-x_1}\leq f'_-(x_2)$$
\textbf{Доказательство:}

Сначала докажу, что $f'_-(x_1)\leq f'_+(x_1)$. Замечу, что $x_1$ в таком случае не должно быть граничной (иначе предела существовать просто не будет). Значит есть какая-то $s$ левее $x_1$ и какое-то $t$ правее $x_1$. Посмотрю на данные выражения:
$\cfrac{f(t)-f(x_1)}{t-x_1}$ и $\cfrac{f(c)-f(x_1)}{c-x_1}$. 

По теореме о трех хордах: $\cfrac{f(c)-f(x_1)}{c-x_1} \leq \cfrac{f(t)-f(x_1)}{t-x_1}$.

Замечу, что при устремлении $s$ к $x_1$,  $\cfrac{f(c)-f(x_1)}{c-x_1}$ будет увеличиваться по теореме о трех хордах (см изобр, напишите т. о трех хордах для $s,s',x_1$).

Замечу, что при устремлении $t$ к $x_1$,  $\cfrac{f(t)-f(x_1)}{t-x_1}$ будет  уменьшаться по теореме о трех хордах (см изобр, напишите т. о трех хордах для $x_1,t',t$).


\begin{center}
 \includegraphics[width = 15cm]{assets/integral_1.png}
\end{center}

Заметим, что первая функция ограничена сверху второй, а вторая ограничена снизу первой. Откуда существуют  $f'_-(x_1), f'_+(x_1)$. Теперь применим теорему о предельном переходе в неравенствах и получим, что $f'_-(x_1)\leq f'_+(x_1)$.

Теперь докажем вторую часть.\\
Возьму $t$ на отрезке $(x_1,x_2)$. Посмотрю на  $\cfrac{f(t)-f(x_1)}{t-x_1}$ и $\cfrac{f(t)-f(x_2)}{t-x_2}$

Заметим, что исходя из этого, тк монотонно возрастает и ограниченна снизу и сверху (по тем же соображениям, что и до этого)
$$\exists  \lim\limits_{t \rightarrow x_1 + 0}\cfrac{f(t)-f(x_1)}{t-x_1} = f'_+(x_1)$$ и тк $\cfrac{f(t)-f(x_1)}{t-x_1}\leq\cfrac{f(x_2)-f(x_1)}{x_2-x_1}$ по лемме о трех хордах, то выполнено второе неравенство.
$$\exists  \lim\limits_{t \rightarrow x_2 - 0}\cfrac{f(t)-f(x_2)}{t-x_2} = f'_-(x_2)$$ и тк $ \cfrac{f(x_2)-f(x_1)}{x_2-x_1}\leq \cfrac{f(t)-f(x_2)}{t-x_2}$ по лемме о трех хордах, то выполнено третье неравенство.

\hfill Q.E.D.

\textbf{Следствие 1.} $f$ - выпукла на $\langle a,b\rangle \Rightarrow f$ непр на $(a,b)$.

\textbf{Следствие 2.} $f$ - выпукла на $\langle a,b\rangle \Rightarrow f$ не дифф. на $(a,b)$ в не более чем счетном множестве (множество точек разрыва НБЧС). Это верно, исходя из того, что значения правосторонних пределов и левосторонних растут (теорема об одностор дифф-ти вып. функции). и берем рациональное число на таком интеравле.

todo: добавить рисунок выпуклая вниз, выпуклая вверх

\thmm{Теорема (выпуклость в терминах касательных)}

$f$ - дифф. на $\langle a,b \rangle$. Тогда

$f$ - вып. вниз $\Leftrightarrow$ График $f$ лежит не ниже любой касательной:
$$\forall x_0,x \in \langle a,b\rangle: f(x) \geq f(x_0) + f'(x_0)(x-x_0)$$
\textbf{Доказательство:}

Докажем в правую сторону. Возьму $x>x_0$, тогда по предыдущей теореме: $f'(x_0)\leq \cfrac{f(x)-f(x_0)}{x-x_0}$. Домножу и победил. Аналогично $x<x_0$.

Докажем в левую сторону. Возьмем 3 точки, $x_1<x_0<x_3$:
$$f(x_3)\geq f(x_0)+f'(x_0)(x_3-x_0),\quad f(x_1)\geq f(x_0)+f'(x_0)(x_1-x_0)$$
$f'(x_0)\leq \cfrac{f(x_3)-f(x_0)}{x_3-x_0}$ и $ \cfrac{f(x_1)-f(x_0)}{x_1-x_0} \leq f'(x_0)$, тогда  по лемме о трех хордах $f$ выпукло.

\hfill Q.E.D.


\deff{def:} Дано множество $A$ выпуклое в $R^2$. Прямая $L$ называется \deff{опорной} к $A$ в точке $x_0$, если $L$ проходит через $x_0$ и множество $A$ лежит в одной полуплоскости (замкнутой).

\thmm{Теорема (дифф. критерий выпуклости)}

1) $f$ - дифф на $(a,b)$, непр на $\langle a,b \rangle$. Тогда $f$ - выпукло на $\langle a,b\rangle \Leftrightarrow f'$ возрастает на $(a,b)$.

2) $f$ непр на $\langle a,b \rangle$, $f$ - дважды дифф на $(a,b)$. Тогда $f$ - вып. $\Leftrightarrow f''\geq 0$ на $(a,b)$.

\textbf{Доказательство:}

1) $\Rightarrow$ очевидно из теоремы об односторонней дифф-ти.\\
$\Leftarrow$ Проверим утверждение леммы о трех хордах. 

$\cfrac{f(x_2)-f(x_1)}{x_2-x_1}= f'(c_1) $ по теореме Лагранжа.
$\cfrac{f(x_2)-f(x_2)}{x_3-x_2}= f'(c_2) $ по теореме Лагранжа.

Так как $c_1 < c_2$, а $f'$ возрастает, то нужное неравенство выполняется.

2) $f$ - выпуклое $\Leftrightarrow$ $f'$ возрастает $\Leftrightarrow$ $(f')'\geq 0 $

\hfill Q.E.D.
\pagebreak


\subsection{Правило Лопиталя.}

\thmm{Лемма (об ускоренной сходимости)}

Пусть даны $f,g: D \in R \rightarrow R$, $a$ - предельная точка $D$ в $\overline{\mathbb{R}}$

Пусть $\exists U(a), f,g \neq 0$ в $U(a)\cap D$ -  выколотой.

$\lim\limits_{x\rightarrow a}f(x) = 0, \lim\limits_{x\rightarrow a}g(x) = 0$. Тогда:
$$\forall (x_n): x_n \rightarrow a, x_n \in D, x_n \neq a,\exists (y_n): y_n \rightarrow a, y_n \in D, y_n \neq a: \lim\limits_{k \rightarrow \infty} \cfrac{f(y_k)}{g(x_k)}<\cfrac{1}{k}\text{ и}\lim\limits_{k \rightarrow \infty} \cfrac{g(y_k)}{g(x_k)}<\cfrac{1}{k}$$
\textbf{Доказательство:}

Давайте будем выбирать такие $y_k$, что:
$$\left|\cfrac{f(y_k)}{g(x_k)}\right|< \cfrac{1}{k} \text{ и }\left|\cfrac{g(y_k)}{g(x_k)}\right|< \cfrac{1}{k}$$

Очевидно, что мы сможем выбрать такие $y_k$. А из этого уже следует то, что нам надо.

\hfill Q.E.D.

\textbf{Замечание:} утверждение верно, если $\lim\limits_{x\rightarrow a}f(x) = +\infty, \lim\limits_{x\rightarrow a}g(x) = +\infty$

\thmm{Теорема(пр. Лопиталя)}

$f,g: (a,b) \rightarrow\mathbb{R}$, дифф $g'\neq 0$ на $(a,b)$

$\cfrac{f'(x)}{g'(x)} \xrightarrow{x \rightarrow a+0} A \in \overline{\mathbb{R}}$. Пусть $\lim\limits_{x\rightarrow a+0}\cfrac{f(x)}{g(x)}$ - неопределенность $\left(\cfrac{0}{0}, \cfrac{\infty}{\infty}\right)$

Тогда: $\lim\limits_{x\rightarrow a+0}\cfrac{f(x)}{g(x)}$

\textbf{Доказательство:} 

Замечание о корректности: тк $g'\neq 0$, то $g$ - строго положительно или отрицательно в какой-то окрестности $a$.

По Гейне. Возьму $(x_n): x_n \rightarrow a,x_n\neq a$. Берем $y_n$ из Лопиталя. 

Теорема Коши: $\cfrac{f(x_k)-f(y_k)}{g(x_k)-g(y_k)} = \cfrac{f'(\xi_k)}{g'(\xi_k)}$, где $\xi_k \in (x_k, y_k)$.

$\cfrac{f(x_k)}{g(x_k)} = \cfrac{f(y_k)}{g(x_k)} + \cfrac{f'(\xi_k)}{g(\xi_k)}\left(1 - \cfrac{g(y_k)}{g(x_k)}\right)$

\pagebreak

\subsection{Определенный интеграл.}


\deff{def:} \deff{Фигура} - это ограниченное подмножество в $R^2$. $\varepsilon$ - множество всех возможных фигур.

$\sigma: \varepsilon \rightarrow [0,+\infty)$ ---  назовем \deff{площадью}, если:
\begin{enumerate}
    \item Аддитивно: $A_1,A_2 \in \varepsilon, A_1\cap A_2 = \emptyset, \sigma(A_1\cup A_2) = \sigma(A_1)+\sigma(A_2)$
\item Нормировка: $\sigma ([a,b]\times[c,d]) = (b-a)(d-c)$.
\end{enumerate}

\textbf{Замечание.} Площади существуют.

\textbf{Замечание.} \begin{enumerate}
    \item Она обладает монотонностью по включению: $A \subset B, \sigma(A) \leq \sigma(B)$, так как: $B = A + (B \setminus A) \Rightarrow \sigma(B) = \sigma(A) + \sigma(B \setminus A)$.
    \item $\sigma(\text{вертик отрезок})=0$, так как его площадь всегда меньше окружающего его прямоугольника с шириной и высотой $= \forall \varepsilon > 0$.
\end{enumerate}

\deff{def:}  $\sigma:\varepsilon\rightarrow [0,+\infty)$ --- \deff{ослабленная площадь}, если выполнено:
\begin{enumerate}
    \item монотонна.
    \item нормирована.
    \item ослабленная аддитивность: Есть $E \in \varepsilon: l $ - вертик. прямая $L^-$ - левая полуплоскость, $L^{+}$ - правая полуплоскость (замкнутая полуплоскость), тогда  $E_1 = E\cap L^-, E_2 = E \cap L^+: \sigma(E)=\sigma(E_1)+\sigma(E_2)$
\end{enumerate}

Пример осл. площади:
\begin{enumerate}
    \item $\sigma(A) = \inf (\sum \sigma(P_k), \text{где $A = \bigcup\limits_{\text{конеч}}P^k, \text{где $P_k$ - прямоугольник}$})$ 
    \item $\sigma(A) = \inf (\sum \sigma(P_k), \text{где} A = \bigcup P^k, \text{где $P_k$ - прямоугольник})$
\end{enumerate}

todo: написать отличие.

\deff{def:} \deff{Срезка} - $f:\langle a,b\rangle \rightarrow \mathbb{R} $.
\begin{enumerate}
    \item \uline{положительная} --- $f^+ = \max(f,0)$
    \item \uline{отрицательная} --- $f^- = \max(-f,0)$
\end{enumerate}

todo: вставить рисунок

\deff{def:} $f:[a,b] \rightarrow \mathbb{R}, f\geq 0$ ПГ $(f,[a,b]) = \{(x,y): x\in [a,b], y \in [0,f(x)]\}$.

\deff{def:} $f: [a,b] \rightarrow \mathbb{R}$, $f$ - непр., $\sigma $- осл. адд площадь, тогда определенный интегралом $f$ по отрезку $[a,b]$ назовем: $$\dint\limits_{a}^b f =\dint\limits_{a}^b f(x) dx =\sigma(\text{ПГ($f^+$,$[a,b]$)})-\sigma(\text{ПГ($f^-$,$[a,b]$)})$$

\textbf{Простейшие свойства:}
\begin{enumerate}
    \item Если $f \ge 0$ на $[a, b]$, тогда $\integral{a}{b}f \ge 0$
    \item Если $f = c$ (константа), тогда $\integral{a}{b}c = c \cdot (a - b)$
    \item $\integral{a}{b}(-f) = -\integral{a}{b}f$
    \item $\integral{a}{a}f = 0$
\end{enumerate}

\textbf{Свойства интеграла:}
\begin{enumerate}
    \item Аддитивность по промежутку: $\forall c \in [a,b]:\dint\limits_{a}^b =\dint\limits_{a}^c + \dint\limits_{c}^b $
    \item Монотонность: $f\leq g$ - непр., то $\dint\limits_{a}^b f(x)\leq \dint\limits_{a}^b g(x)$. 

    Говорят: Проинтегрируем неравенство $f\leq g$, на отрезке $[a,b]$.

    \item $(b-a)\min_{[a,b]} f \leq\integral{a}{b}f\leq (b-a)\max_{[a,b]}f$

    Делается с помощью монотонности и интегрирования $\min_{[a,b]}f\leq f\leq\max_{[a,b]}$

    \item $\left|\integral{a}{b}f(x)dx\right|\leq \integral{a}{b}|f(x)|dx$

    Проинтегрируем $-|f|\leq f\leq |f|$ и получим то, что хотим.

    \item \thmm{Теорема о среднем}

    Функция $f \in C([a,b])$. Тогда $\exists c \in [a,b]$, что:
    $$\integral{a}{b}f = f(c)(b-a)$$
    \textbf{Доказательство:} 
    
    $a=b$ - скучно. Если $a\neq b$, напишем неравенство п.3:
    $$\min f\leq \cfrac{1}{b-a}\integral{a}{b}f\leq \max f$$ А мы знаем, что функция непрерывна, тогда по теореме о промежуточном значении:
    
    $$\exists \, c  = \cfrac{1}{b-a}\integral{a}{b}f$$. 
    
    \hfill Q.E.D.
\end{enumerate}

\deff{Интеграл с переменным верхним пределом} -  $\Phi:[a,b] \rightarrow R: \Phi(x)= \integral{a}{x}f$

\deff{Интеграл с переменным нижним пределом} -  $\psi:[a,b] \rightarrow R: \psi(x)= \integral{x}{b}f$

для $f\in C([a,b])$.

\thmm{Теорема (Барроу)}

В усл. определений. Доказать, что $\Phi$ дифф на $[a,b]$, $\Phi'(x)=f(x)$.

\textbf{Доказательство:}

$y> x: \lim\limits_{y \rightarrow x+0} \cfrac{\Phi(y)-\Phi(x)}{y-x} = \lim\limits_{y \rightarrow x+0} \cfrac{\int_a^y f - \int_a^x f}{y-x} =\lim\limits_{y \rightarrow x+0} \cfrac{1}{y-x} \integral{x}{y}f =\lim_{y\rightarrow x+0} f(c)$, \\
где $c$ лежит между $x,y$ из теоремы о среднем.

Получим, что правосторонняя производна равна $f(x)$. Аналогично про левостороннюю. Откуда производная это $f(x)$.

 \hfill Q.E.D.
 
\textbf{Замечание} Мы построили первообразную для функции $f$.

\thmm{Теорема (формула Ньютона-Лейбница)}

$f\in C([a,b])$, $F$ - первообразная $f$ на $[a,b]$. Тогда

$\integral{a}{b}f(x)dx = F(b)-F(a) = F(x)\big|^b_a$

\textbf{Доказательство:}

$F = \Phi + c$, по теореме 2. Поэтому сделаем некоторые преобразования:
$$\integral{a}{b}f(x)dx = \Phi(b)= \Phi(b)-\Phi(a) = (F(b)-c)-(F(a)-c)=F(b)-F(a)$$
 \hfill Q.E.D.

 \textbf{Следствие:} Этот определенный интеграл не зависит от выбора $\sigma$. 
 
 \textbf{Замечание:} Откажемся от соглашения $a\leq b$ и введем для $d<c:$
 $$\integral{c}{d}= - \integral{d}{c} = F(d) - F(c)$$
 \thmm{Микротеорема (Линейность интеграла)}

 Для $f,g \in C([a,b])$, $\alpha,\beta \in \mathbb{R}$, выполнено:
$$\integral{a}{b}\alpha f+ \beta g = \alpha \integral{a}{b}f + \beta\integral{a}{b}g$$
\textbf{Доказательство:}

$(\alpha F+\beta G)\big|^b_a = \alpha F\big|^b_a + \beta F|^b_a$ из линейности неопредел. интеграла.
 \hfill Q.E.D.

 \thmm{Теорема (Интегрирование по частям)}

 $f,g \in C^1([a,b])$. Тогда $\integral{a}{b}fg' = fg\big|_a^b-\integral{a}{b}f'g$

 \textbf{Доказательство:}

Из теоремы  о свойствах неопределенного интеграла:
$$fg = \text{првобр} (fg'+f'g) \Rightarrow \integral{a}{b}(fg' + f'g) = fg\big|_{a}^b$$
 \hfill Q.E.D.
 
 \thmm{Теорема (о замене переменных)}

$f \in C(\langle a,b\rangle), \,\, \varphi\langle\alpha, \beta\rangle \rightarrow \langle a, b \rangle, \,\, \varphi \in C^1, \,\, [p,q] \subset \langle \alpha,\beta\rangle$. Тогда:

$$\integral{p}{q}f(\varphi(t))\varphi'(t)dt = \integral{\varphi(p)}{\varphi(q)}f(x)dx$$

\textbf{Доказательство:}

$F$ - первообразная $f$, тогда $F(\varphi(t))$ - первообразная $f(\varphi(t))\varphi'(t)$ и все получается.
 
 \hfill Q.E.D.

1:09 мат анализ кохась лекция 2. Я ничего не понял про нижние два замечания


 \textbf{Замечание.} Может показаться, что множество $\varphi([p,q])$ шире $[\varphi(p),\varphi(q)]$. 

 \textbf{Замечание.} Может быть, что $\varphi(p) > \varphi(q)$

$I_f$ - среднее значение $f$  на $[a,b]$  $\cfrac{1}{b-a}\integral{a}{b}f$
 
 \thmm{Теорема (Неравенство Чебышёва)}

 $f,g \in C([a,b])$ обе возрастают. Тогда $I_f \cdot I_g \leq I_{fg}$, то есть
$$\integral{a}{b}f \cdot \integral{a}{b}g\leq (b-a)\integral{a}{b}fg$$
\textbf{Доказательство:}

Тк функции возрастают, то $\forall x,y \in [a,b]: (f(x)-f(y))(g(x) - g(y))\geq 0$.
$$f(x)g(x)-f(y)g(x)-f(x)g(y) + f(y)g(y) \geq 0$$ 
Давайте зафиксируем $y$ и проинтегрируем по $x$ и поделю на $b-a$. Получу:
$$I_{fg} - f(y) I_g - I_f g(y) + f(y)g(y)\geq 0 $$
Давайте зафиксируем $x$ и проинтегрируем по $y$ и поделю на $b-a$. Получу:
$$I_{fg}-I_fI_g -I_gI_f + I_{fg} \geq 0$$
\hfill Q.E.D.

\textbf{Пример (Ш. Эрмит)} \\
Пусть мы хотим посчитать $H_n = \cfrac{1}{n!}\integral{-\frac{\pi}{2}}{\frac{\pi}{2}} \left(\cfrac{\pi^2}{4}-t^2\right)^n \cos t dt$\\
$\integral{a}{b}fg' = fg\Big|^a_b - \integral{a}{b}f'g$. Воспользуюсь этим в дальнейших рассуждениях\\
$H_n =\left [ \begin{array}{l}
    f = \left(\cfrac{\pi^2}{4}-t^2\right)^n\\  
     g = \cos t 
\end{array}\right] =\left(\cfrac{\pi^2}{4}-t^2\right)^n \sin t\Bigg|^{\pi/2}_{\pi/2} + \cfrac{2}{(n-1)!}\integral{-\pi/2}{\pi/2}t(\cfrac{\pi^2}{4}-t^2)^{n-1}\sin t dt $

я не хочу это писать 1:30 2 лекция

$H_n = (4n-2)H_{n-1} - \pi^2H_{n-2} = P(\pi^2)$ - многочлен, от $\pi^2$, где $\deg P\leq n$.

\thmm{Теорема (Пи иррационально)}

$\pi$ - иррационально. Проверим, что $\pi^2$ иррационально. 

\textbf{Доказательство:}

Пусть $\pi^2 = \cfrac{p}{q}$. Тогда $q^nH_n=\cfrac{q^n}{n!}\integral{-\pi/2}{\pi/2}\left(\cfrac{\pi^2}{4}-t^2\right)^n \cos t \,\, dt = q^n P(\pi^2)$ - целое число. А слева неотрицательная функция.

$0<q^nH_n \leq \cfrac{q^n}{n!}4^n\pi=\cfrac{(4q)^n}{n!}\pi\rightarrow 0$, но с другой стороны, оно должно быть целым. Противоречие.

\hfill Q.E.D.

\deff{def:} $f:[a.b] \rightarrow \mathbb{R}$ \deff{кусочно-непрерывной.}

$\exists A \ = \{x_1,\ldots,x_n\} \subset [a,b]$. Такая функция будет непрерывны на $[a,b]$, кроме этих точек, а в них происходят скачки.


\deff{def:} $F:[a,b] \rightarrow \mathbb{R}$ - \deff{почти первообразная} функции f.

$F$ - непр и $\exists A = \{x_1,\ldots,x_n\} \subset[a,b]$.
$\forall x \in [a,b] \textbackslash A: \exists F'(x) = f(x)$ и $\forall x \in A: \exists F'_+(x),F'_-(x)$


$f$ - кусочно-непрерывно на $[a,b]$. $x_0 = a, x_n = b$. Положим $\integral{a}{b} = \sum\limits_{i=1}^{n+1}\integral{x_{i-1}}{x_i}f\Big|_{[x_{i-1},x_i]}$

\textbf{Утверждение.} Если $f$ - кусочно - непрерывна тогда: $\integral{a}{b}f=F(b)-F(a)$.

Утверждение очевидно по определению.

\textbf{Следствие:} Все теоремы, использующие в доказательство только формулу Ньютона-Лейбница у нас уже доказаны!

\textbf{Пример (Неравенство Чебышева для сумм)}

$a_1\leq \ldots \leq a_n, b_1\leq \ldots \leq b_n$.

Тогда $\displaystyle\left(\cfrac{1}{n} \sum\limits_{i=1}^na_ib_i\right)\geq \left( \cfrac{1}{n}\sum a_i\right)\left(\cfrac{1}{n}\sum b_i\right)$

\textbf{Доказательство:}
\begin{center}
   \includegraphics[width = 19 cm]{assets/integral_2.png}
\end{center}
Возьмем доску Константина Петровича для лучшего понимания. Давайте возьмем две функции $f(x)$, $g(x)$, как показано на рисунке. Вспомним, что у нас есть неравенство Чебышева, которое записано на правой стороне доски. Тогда очевидно подстановкой в него наших $f(x),g(x)$ и $a=0,b=n$, мы получим нужное неравенство.
\hfill Q.E.D.

\pagebreak
\subsection{Приложение к определенным интегралам.}

Введем некоторые обозначения:

$\Segm ([a,b])$ - множество всевозможных отрезков, лежащих в $[a,b]$

$\varPhi: \Segm([a,b]) \rightarrow \mathbb{R}$ - \deff{функция для промежутка}.

Введем \deff{аддитивные функции для промежутка}:
$$\forall [p,q] \in \Segm[a,b]: \forall c \in (p,q): \varPhi([p,c]) + \varPhi(c,q)= \varPhi([p,q])$$

\deff{def:} $f: \langle a,b \rangle \rightarrow \mathbb{R}$, $\varPhi: \Segm(\langle a,b \rangle) \rightarrow \mathbb{R}$ - а.ф.п:

$f$ - \deff{плотность} $\varPhi$: $\forall \Delta \in \Segm (\langle a,b\rangle):$$inf_{\Delta} f \cdot len(\Delta)\leq \varPhi(\Delta)\leq sup_{\Delta} f \cdot len(\Delta)$

\thmm{Теорема(о вычисл. а.ф.п.по ее плотности)}

Дана плотность
$f:\langle a,b\rangle \rightarrow \mathbb{R}$, $\varPhi: \Segm(\langle a,b\rangle)$ - а.ф.п., $f$ - непр.

Тогда $\forall \Delta \in \Segm(\langle a,b\rangle) $, $\varPhi(\Delta) = \integral{\Delta}{}f$

\textbf{Доказательство:}

Н.У.О. считаем, что $\Delta = [a,b]$. Тогда возьмем $F(x)$, такую что:
$$F(x) = \left[ 
      \begin{gathered} 
        0, x=a \\ 
        \varPhi([a,x]), x \in (a,b) \\ 
      \end{gathered} 
\right.$$
Проверим, что $F$ - первообразная f:

$F'_{+}(x) = \cfrac{F(x+h)-F(h)}{h} = \cfrac{\varPhi([a, x +h])-\Phi([a,x])}{h} = \cfrac{\varPhi(x,x+h)}{h}\in [\min f, \max f]$ на промежутке $x+x_0$ из ее плотности  

Получили, что правосторонняя производная $f$ и левостороняя производные существуют.

\hfill Q.E.D.



\uline{\textbf{Пример: Площадь криволинейного сектора.} }

$[a,b] \subset [0,2\pi)$

$\rho:[a,b]\rightarrow \mathbb{R}, \rho>0$

$\varphi \in [a,b] \rightarrow (\varphi,\rho(\varphi))$

Введем определение: \deff{Сектор} $[\alpha,\beta] = \{(\varphi,r)\subset R^2: \varphi\in[\alpha,b], 0\leq r\leq p(\varphi)\}$

$\varPhi: \Delta = \sigma(\text{Сектора}) $, $\Delta \in \Segm([a,b])$


\thmm{Теорема.}  

В указанных условия, а так же $\rho: [a,b] \rightarrow \mathbb{R}, \rho>0$ и непрерывна. $[\alpha,\beta]\in \Segm([a,b])$. Тогда: $$\varPhi([\alpha,\beta]) = \cfrac{1}{2}\integral{\alpha}{\beta}\rho^2(\varphi) \, d\varphi$$
\textbf{Доказательство:}

Если мы докажем, что $\frac{1}{2}\rho^2(\varphi)$ - плотность $\varPhi$, тогда по предыдущей теореме, мы получим, что данная формула будет верна. Будем опрделять определение плотности.

$\Delta = [\alpha,\beta]$, откуда Сектор$[\alpha,\beta]\subset$ Криволинейного вектора($O, \max \rho, [\alpha,\beta]$).

Криволинейный вектор в в данном случае подразумевает сектор окружности, нарисованный на чертеже. Так же на нем вы видите серым - Сектор$[\alpha,\beta]$.

\begin{center}
   \includegraphics[width = 10 cm]{assets/integral_3.png}
\end{center}

Как мы знаем из геометрии: площадь сектора окружности = $\cfrac{\alpha}{2}R^2$.

Откуда из монотонности площади:
$$\varPhi([\alpha, \beta])\leq \sigma(\text{Крив. вектор}) = \cfrac{1}{2} (\beta-\alpha) (\max\limits_{[a,b]}\rho)^2$$
Аналогично можно оценить нижним сектором. То есть:
$$\cfrac{1}{2} (\beta-\alpha) (\min\limits_{[a,b]}\rho)^2\leq\varPhi([\alpha, \beta])\leq \cfrac{1}{2} (\beta-\alpha) (\max\limits_{[a,b]}\rho)^2$$
Откуда это и правда плотность, поэтому верно.

\hfill Q.E.D.

\textbf{Кохась: хочу эксперимент}

$\sigma (\text{ПГ($f, [a,b]$)}) = \integral{a}{b}f\, dx$, где $f \geq 0, f$ - непрерывно.
$\gamma(t) =  \left (\begin{gathered}
    x = x(t)\\
    y(x) = y(t)
\end{gathered}\right)$, $\gamma:[p,q] \rightarrow R^2$.
\begin{center}
   \includegraphics[width = 7 cm]{assets/integral_4.png}
\end{center}
\textbf{Замечание от Славы}: вообще $x(t)$ должно монотонно возрастать, иначе странные загагулины будут давать одну и ту же площадь, но КПК про это ничего не сказал.

Причем $\gamma$ -  гладкое изоображение (дифференцируема столько раз сколько надо).

Получилась какая-то кривая (как на рисуночке сверху), и я хочу смотреть подграфики такой кривой. Тогда:
$$\sigma A = \integral{a}{b}"y(x)"dx = \left[\begin{gathered}
    x = x(t)\\
    y = y(t)
\end{gathered}\right] = \integral{p}{q}y(t) x'(t) dt$$
Теперь мы умеем вычислять интегралы не только в декартовых координатах.

todo: вставить 2 формулы 1.50

\textbf{Пример:}

$\begin{cases}
    x(t) = r(t-\sin(t)), r\in R\\
    y(t) = r(1-\cos(t)), t \in[0,2\pi]
\end{cases}$ - путь, который описывается данной формулой - \deff{циклоид}.

Фиксируем точку в нуле и катим окружность по нашему полю. Мы знаем, что $x$ монотоннен
\begin{center}
   \includegraphics[width = 13 cm]{assets/integral_5.png}
\end{center}
И теперь я хочу найти площадь серого подграфика:

$S  = \integral{0}{2\pi r}y(x)dx  = \left [ \begin{gathered}
    x(t) = r(t-\sin t) \\
    y(t) = r(1-\cos t)
\end{gathered}\right] = \integral{0}{2\pi}r^2(1-\cos t)^2 dt = r^2\integral{0}{2\pi}(1-\cos t)^2 = 3 \pi r^2$


\textbf{Пример (Изопометрическое неравентсво.)}

$G \subset \mathbb{R}^2$ - выпукло, замкнуто, ограниченно. 

Пусть $diam(G) = \sup\limits_{a,b\in G} (\rho(a,b))$ - диаметр. $diam(G) = d$. Тогда: $\sigma(G)\leq \cfrac{\pi}{4}d^2$

todo: тут во-первых скипнут рисунок, во-вторых я не осознал 2:20

$\rho(\varphi) = \max (r: (\varphi,r)_{max} \in G)$ -  непрерывна.

Упражнение: доказать непрерывность (возможно спросят на экзамене)

$\overline{\varphi} = \varphi + \cfrac{\pi}{2}$

$$\sigma(G)= \frac{1}{2}\integral{-\frac{\pi}{2}}{\frac{\pi}{2}}\rho^2(\varphi)\,d\varphi = \frac{1}{2}\left(\integral{0}{\frac{\pi}{2}} + \integral{\frac{-\pi}{2}}{0}\right) = \frac{1}{2}\integral{\frac{-\pi}{2}}{0}\rho^2(\varphi)\, d \varphi  + \cfrac{1}{2}\integral{0}{\frac{\pi}{2}}\rho^2\left(\overline{\varphi}-\cfrac{\pi}{2}\right)\,
d\overline{\varphi}=$$

$$=\frac{1}{2}\integral{\frac{\pi}{2}}{0}\rho^2(\varphi) + \rho^2(\varphi - \frac{\pi}{2}) d\varphi = \cfrac{1}{2}\integral{0}{\frac{\pi}{2}}"AB^2" d\varphi \leq \cfrac{1}{2}\integral{0}{\frac{\pi}{2}}d^2 d\varphi = \frac{d^2\pi}{4}$$
