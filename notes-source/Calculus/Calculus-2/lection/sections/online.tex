Прошлая лекция:

$\sum\limits_{n=1}^{+\infty} a_n = S$

$S  = \lim\limits_{N \xrightarrow{} +\infty} (\sum\limits_{n=1}^N
 a_n )$


% Кохась говорит про признаки и начинает лекцию

\thmm{Теорема (Интегральный признак Коши)}

Пусть у вас есть функция $f$ - непрерывная на $[1,+\infty)$, монотонна, $f\geq 0 $. Тогда $\sum\limits_{n=1}^{+\infty}f(n)$ и $\integral{1}{+\infty}f(x)dx$ сходятся и расходятся одновременно.

\textbf{Доказательство:}

\textbf{Замечание от Славы.} Как будто очень похоже на то, что слева написана сумма Римана.

Существует $\exists \lim\limits_{x\rightarrow +\infty}f(x) = A \geq 0$.

Еcли $A>0$, то очевидно расходится. Если $A = 0$, тогда из прошлых лемм у нас есть, частичные суммы, а интеграл замкнут между такими суммами:
$$\sum\limits_{n=2}^{N}f(n)\leq \integral{1}{N} f(x) dx\leq \sum\limits_{n=1}^{N} f(n)$$

%todo: вставить рисунок

\hfill Q.E.D.

\textbf{Пример:} 
$$\sum\limits_{n=2}^{+\infty} \cfrac{1}{n^p (\ln n)^q}$$
Как мы показывали ранее мы знаем, когда соотв. интеграл сходится и расходится.

\textbf{НО НЕЛЬЗЯ ЗАБЫВАТЬ} О ТОМ, ЧТО ФУНКЦИЯ ДОЛЖНА  БЫТЬ МОНОТОННА

\deff{Абсолютная сходимость.}

$a_n$ - любого знака. $\sum\limits_{n=1}^{+\infty}$ --- абсолютная сходимость если:
\begin{enumerate}
    \item $\sum a_n$ - сходится
    \item $\sum |a_n|$ - сходится
\end{enumerate}

\textbf{Пример:}
$$\cfrac{1}{1+x^2}\leq 1 - x^2  + x^4 - \ldots (-1)^N x^{2N} + \cfrac{(-1)^{N+1}x^{2N+2}}{1+x^2}$$
Проинтегрируем по $[0,1]$, получим:
$$\cfrac{\pi}{4}= 1 -\cfrac{1}{3} + \cfrac{1}{5}+\ldots +\cfrac{(-1)^N}{2N+1}+\integral{0}{1}\cfrac{(-1)^{N+1}x^{2N+2}}{1+x^2}dx$$
Интеграл справа по модулю $\leq$ %todo 28

Ряд $\sum\limits_{n=0}^{+\infty}\cfrac{(-1)^n}{2n+1}$ - не сходим абсолютно

%todo: 30 минута лекции прослушать что тут говорил кохась и выписаьб

Такая формула называется суммой \deff{Грегори-Лейбница}. %todo: мб подумать

\thmm{Теорема.}

$a_n$ --- любого знака. Тогда эквивалентно:
\begin{enumerate}
    \item $\sum\limits_{}a_n$ - абсолютная сходимость.
    \item $\sum\limits_{}|a_n|$ - сходится.
    \item $\sum a_n^+, \sum a_n^-$ - оба сходятся. Где $a_n^+ = \max(a_n,0), a^-_n= \max(-a_n,0)$
\end{enumerate}

\textbf{Доказательство:} смотри теорему в интегралах

\pagebreak
\subsection{Сходимость рядом с произвольными знаками слагаемых}

\thmm{{Теорема (Признак Лейбница)}}

$c_1 \geq c_2 \geq \ldots \geq 0$ (т.е монотонность). Пусть $c_n \rightarrow 0 $. Тогда $\sum\limits_{n=1}^{+\infty} (-1)^{n+1} c_n$ - сходится.

\textbf{Доказательство:}

%37 минут рисунок

И давайте все синие квадратики подвинем налево. Тогда мы получим, что такая сумма будет ограничена. Но мы доказали, что $(c_1-c_2) +(c_3-c_4) + \ldots$  сходится.  Осталось проверить нечетные частичные суммы. $S_{2n+1} =S_{2n}+ c_{2n+1}$ и именно если $c\rightarrow 0 $, то $S_{2n+1}$ стремится к тому же и мы победили.

\textbf{Более формальное доказательство}:

Пусть $S_{2k}  = c_1-c_2 + \ldots +c_{2k-1} - c_{2k}$. Тогда посмотрим на четные суммы:

\begin{enumerate}
    \item $S_{2k}\leq S_{2k+2}$, тк добавили что-то неотрицательное.
    \item $S_{2k}\leq c_1: S_{2k} = c_1-(c_2-c_3)-(c_4-c_5) - (c_{2k-2}-c_{2k-1})-c_{2k}\leq c_1$
\end{enumerate}

Значит существует предел $S_{2k}$. И используйте концовку прошлой.

\hfill Q.E.D.

\thmm{Секретное приложение к признаку Лейбница}

$c_1\geq c_2 \geq \ldots \geq 0  $, $c_n\rightarrow 0 $

$|\sum\limits_{n=N}^{+\infty} (-1)^{n+1}c_n|\leq c_N$

\textbf{Доказательство:} см. теорему выше.

\textbf{Пример:}
\begin{enumerate}
    \item $\sum \cfrac{(-1)^n}{n + \sin n}$ сходится по признаку Лейбница
    \item $\sum\limits_{n=2}^{+\infty}\cfrac{(-1)^n}{n +(-1)^n}$ этот ряд не удовл. признаку Лейбница, тк не монотонна 
\end{enumerate}

\textbf{Очень грустная картинка.}
%todo: вставить грустную картинку   1.04

\deff{Преобразование Абеля (суммирование по частям)}.

$\sum\limits_{n=1}^{N} a_n b_n = A_N b_N + \sum\limits_{n=1}^{N-1}A_n(b_n-b_{n+1})$, где $A_n = a_1 + \ldots + a_n$

\textbf{Доказательство:} Раскройте сумму и получите magic. (не забудьте проверить края)

\thmm{Теорема (признак Абеля и Дирихле)}

\begin{enumerate}
    \item
    \begin{enumerate}
        \item  Пусть частичные суммы последовательности $a_n$ - ограниченны: $\exists C_A: \forall n: a_1 + \ldots + a_n \leq C_A$.
         \item Пусть $b_n$ - монотонна, $b_n\rightarrow 0 $
    
    \end{enumerate}
     Тогда $\sum\limits_{n=1}^{+\infty}a_nb_n$ - сходится
  
   \item
   \begin{enumerate}
       \item Ряд $\sum a_n$ - сходится.
        \item $b_n$ - монотонна и ограничена. $\exists C_B: \forall $
   \end{enumerate}
    Тогда $\sum\limits_{n=1}^{+\infty}a_nb_n$ - сходится.
   
\end{enumerate}

\textbf{Доказательство:}

\begin{enumerate}
    \item $\sum\limits_{n=1}^N a_n b_n =A_n b_n + \sum\limits_{n=1}^{N-1} A(b_n-b_{n+1})$

$A_N$ - ограниченная и $b_N$ - бесконечно малая.

Ряд $\sum\limits_{n=1}^{+\infty} A_n (b_n-b_{n+1})$ - сходится, потому что он сходится абсолютно. А абсолютно он сходится, потому что:
$$\sum\limits_{n=1}^N |A_n||b_n-b_{n+1}|\leq C_A \sum|b_n-b_{n+1}|$$
- разности под модулем одного и того же знака, поэтому 
$$= C_A|b_1-b_{N+1}|\leq C_A \cdot 2 C_B$$
\item
$\exists$ кон. $\lim\limits_{n\rightarrow + \infty} b_n =\beta$. Разложу ряд и получу: $$\sum\limits_{n=1}^{+\infty}a_n b_n = \sum\limits_{n=1}^{+\infty} a_n \beta + \sum\limits_{n=1}^{+\infty}a_n(b_n-\beta)$$
Правильна ли формула? Не всегда, только если пределы есть.

Заметим, что $\sum\limits_{n=1}^{+\infty}a_n \beta$ сходится по усл. $1$. А вторая сумма сходится по признаку Дирихле (первому пункту нашей теоеремы). Откуда имметт предел и мы победили.
\end{enumerate}

\hfill Q.E.D.

\textbf{Пример:}
$$\sum\limits_{n=1}^{+\infty}\cfrac{\sin n}{n^{\alpha}}, \alpha >0$$
$$|\sin 1 + \sin 2+ \ldots + \sin n| = |\Im (e^i + e^{2i} + \ldots + e^{ni})| \leq \Im |e^i\cfrac{e^{ni}-1}{e^i-1}| \leq \cfrac{2}{e^{i}-1} = C_A$$
Откуда ограничены частичные суммы и $b_n = \cfrac{1}{n^{\alpha}}$ монотонна и $b_n\rightarrow 0 $, то выполнен признак Дирихле, откуда победили.

Пример: смерть монстра  %2.10 примерно

\pagebreak
\subsection{Свойства сходящихся рядов.}

Сюжет $I$ --- группировка слагаемых.

\textbf{3 прикола:}
$$1- 1 + 1 -\ldots \rightarrow ???$$
$$(1-1)+(1-1) + \ldots \rightarrow 0$$
$$1 + (-1 + 1) + \ldots \rightarrow 1$$

Так что группировка если работает, то работает очень хитро.

$\sum a_k = (a_1 + \ldots + a_{n_1}) + (a_{n_1+1}+\ldots + a_{n_2})+\ldots$. И теперь я каждую скобоку заменю на $b_i$. 

\thmm{Теорема}

Используя обозначения выше:
\begin{enumerate}
    \item $\sum a_k$ - сходится. Тогда $\sum b_k$ - сходится  и имеет ту же сумму.
    \item $\forall k: a_k\geq 0$, то $\sum a_k, \sum b_k$ имеют одинаковые суммы (или одновременно расходятся)
\end{enumerate}

\textbf{Доказательство:}

$S_k^{(b)} = S^{(a)}_{n_k}$


\hfill Q.E.D.









