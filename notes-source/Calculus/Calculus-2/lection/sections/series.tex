\subsection{Определения.}

\deff{def:} Пусть дана вещ. последовательность $(a_n)$. 

Выражение вида $S_N  = \sum\limits_{n=1}^N a_n$ --- \deff{частичная сумма ряда}.

Если $\exists \lim\limits_{N \rightarrow +\infty}S_n = L \in \overline{\mathbb{R}}$, то говорят, что $L$ - \deff{сумма ряда} $\sum\limits_{n=1}^{+\infty}a_n$.

В случае $L$ конечного будем называть ряд \deff{сходящимся}. В случае $L = \infty$ или не существования предела реда, будем называть ряд \deff{расходящимся}.  

\textbf{Замечание.}  $a_n = S_n - S_{n-1}$.

\textbf{Примеры:}
\begin{enumerate}
    \item $(a_n): a_n \equiv 0$, то сумма $0$ - сходится
    \item $(a_n): a_n \equiv 1$, то сумма $+\infty$ - расходится
    \item $(a_n): a_n = (-1)^n$, то предела нет и расходится.
    \item $a_n = q^n$. $S_N = 1+\ldots + q^N =\cfrac{q^{N+1}-1}{q-1}$.
    
    Заметим, что это будет сходиться при $q<1$ и $\xrightarrow{n \rightarrow +\infty} \cfrac{1}{1-q}$. 
\end{enumerate}

$\sum\limits_{n=k}^{+\infty} a_n$ - $k$-ый остаток ряда.

\textbf{Свойства рядов:}

\begin{enumerate}
    \item $\sum a_n , \sum b_n$ - сх. $c_n = a_n + b_n$.

        Тогда $\sum c_n$ - сходится и $\sum c_n = \sum a_n + \sum b_n$
    \item $\sum a_n$  - сходится $\lambda\in \mathbb{R} \Rightarrow \sum\lambda a_n$ - сходится и $\sum \lambda a_n = \lambda \sum a_n$.
    \item $\sum a_n$ - сходится, то любой остаток ряда сходится
    \item Если какой-нибудь остаток ряда сходится, то ряд сходится
    \item Ряд сходится $\Leftrightarrow r_n \rightarrow 0 $.
\end{enumerate}

\thmm{Теорема (грабли) (необходимое условие сходимости)}

$\sum a_n$ - сходится. Тогда $a_n \rightarrow 0$

\textbf{Доказательство:}

Да если бы камень умел думать, да если бы он не думал, он бы сходу сделал доказательство.

\hfill Q.E.D.

\textbf{Замечание.} В ОБРАТНУЮ СТОРОНУ НЕ РАБОТАЕТ!!!

\thmm{Теорема (критерий Больцано-Коши)}




