\subsubsection{Задача 1.}

$V_3$ --- пространство геометрических векторов. Отображение $f:V_3 \xrightarrow{} \mathbb{R}$ определено равенством $\forall \overline{x} \in V_3, f(\overline{x})=(\overline{x},\overline{a})$, где $\overline{a} = \overline{i} + 2 \overline{j} - 3 \overline{k}$.

\begin{enumerate}
    \item Доказать, что $f \in V_3^*$
    \item найти коэффициенты $f$ относительно стандартного базиса пространства $V_3$.
\end{enumerate}

\textbf{Решение:}

Ну давайте, докажем, что это линейная форма.
$$\forall x_1,x_2 \in V_3, \lambda \in R:f(x_1 + \lambda x_2) = (x_1+\lambda x_2, a) = (x_1,a) + \lambda(x_2,a) = f(x_1)+\lambda f(x_2)$$
Откуда линейная форма.

Теперь найдем коэффициенты $f$ относительного стандартного базиса. Для этого мы должны применить функцию. К базисным векторам:

$f(e_1) = 1, f(e_2 ) =2, f(e_3) = -3$, то есть $a_f=(1,2,-3)$ --- в стандартном базисе

\subsubsection{Задача 2.}

$P_n$ - пространство многочленов степени не выше $n$. Отображение $f:P_n\xrightarrow{} \mathbb{R}$ определено равенством $\forall p \in P_n: f(p) = p(t_0)$, где $t_0$ - фиксированное константа из $\mathbb{R}$.

\begin{enumerate}
    \item Доказать, что $f\in P_n^*$
    \item Найти коэффициенты $f$ относительно канонического базиса пространства $P_n$
    \item Найти коэффициенты $f$ относительно базиса $1,(t-t_0)$ и так далее.
\end{enumerate}

\textbf{Решение:}

Показать, что это линейная форма крайне тривиально. Подставляя канонический базис мы получим $a_f =(1,t_0,\ldots,t^n)$, подставляя сдвинутый базис получим $a_f = (1,0,0,0,\ldots ,0)$

\subsubsection{Задача 3.}

$e_1,e_2,e_3$ - базис линейного пространства $V$. $\forall x = x^i e_i \in V: f(x)= x^1 + 2x^2 + 3x^3$. Найти выражение для $f$ в базисе $e_1' = e_1 + e_2, e'_2 = e_2 + e_3, e_3' = e_3 + e_1$.

\textbf{Решение:}

Тут можно поступать разными образами. Можно просто подставить в функцию новые базисы и найти их значения. Можно найти обратную матрицу перехода и по ней получить новые значения. В общем тривиально, думать не хочу.

\subsubsection{Задача 4.}

$P_2$ - пространство многочленов степени не выше 2.
\begin{enumerate}
    \item линейная форма $\delta$ сопоставляет каждому многочлену его свободный член. Разложить $\delta$ в комбинацию линейных форм $f^1,f^2,f^3$, где $f^j$ определены равенством $\forall p \in P_2, f^j(p) = p(j)$
    \item Для базиса $f^1,f^2,f^3$ построить сопряженный к нему и с помощью найти координаты $\delta$ в базисе $f^1,f^2,f^3$.
\end{enumerate}

\textbf{Решение:}

Найдем каждую $f$ в базисе $e_1,e_2,e_3$. Для этого мы должны подставить в $f$-ки наши базисные вектора. Получим:

$a_{f^1} =(1, 1,1); a_{f^2}= (1,2,4); a_{f^3}=(1,3,9)$. Получили вот такую штучку. Теперь надо с помощью них собрать $a_{\delta} = (1,0,0)$. Для этого можно решить уравнение но я бы перешел дальше.

$S = \begin{pmatrix}
    1 &1&1 \\
    1&2&4\\
    1&3&9
\end{pmatrix}$. Тогда найдем $T = S^{-1}= \begin{pmatrix}
    3 & -3 & 1 \\
    -2,5 & 4 & -1,5\\
    0,5 & -1 & 0,5
\end{pmatrix}$. Теперь, чтобы найти координаты в базисе $f$ мы должны:
$$a' = a T = (1,0,0) \begin{pmatrix}
    3 & -3 & 1 \\
    -2,5 & 4 & -1,5\\
    0,5 & -1 & 0,5
\end{pmatrix} = (3,-3,1)$$
Все верно!!!
