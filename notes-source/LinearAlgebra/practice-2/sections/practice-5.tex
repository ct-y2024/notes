\subsubsection{Задание 1.}

$e_1 = \begin{pmatrix}
    2 \\
    6\\
    5
\end{pmatrix} ,e_2 =\begin{pmatrix}
    5 \\
    3\\
    7
\end{pmatrix}, e_3 = \begin{pmatrix}
    7 \\
    4\\
    -3
\end{pmatrix}$ - базис, координаты векторов которого заданы относительно некоторого о.н.б. Найти взаимный базис.

\textbf{Решение:}

$\Gamma = E$, по формуле $e^* = e\Gamma_e^{-1}$. Как мы знаем по формуле:

$\Gamma_e = T^T \Gamma T = T^T T$. $\Gamma = \begin{pmatrix}
    65 & 18 & 23\\
    18 & 38 & 53\\
    23 & 53 & 74
\end{pmatrix}$

Далее находим $\Gamma^{-1}$ и по формуле находим взаимный базис.

\subsubsection{Задание 2.}

Тензор $\alpha \in T(2,0)$ задан матрицей $A = \begin{pmatrix}
     0 & 1& 3\\
     2 &3&4\\
     3 &5&2\\
\end{pmatrix}$ в евклидовом пространстве с ковариантным метрическим тензором $\Gamma = \begin{pmatrix}
    21  & -10 & -4\\
    -10 & 5 & 2\\
    -4 & 2 & 1
\end{pmatrix}$.

Найти матрицу тензора:

\begin{enumerate}
    \item с поднятым 1-ым индексом
    \item с поднятным 2-ым индесом.
    \item с поднятыми двумя индексами
\end{enumerate}

\textbf{Решение:}

Найдем контрвариантный тензор $\Gamma^{-1} = \begin{pmatrix}
1 & 2 & 0 \\
2 & 5 & -2 \\
0 & -2 & 5
\end{pmatrix}$

$\beta = \alpha_j^i = \alpha_{\ae j}g^{\ae i} \Rightarrow \beta = \Gamma^{-1} A \in T(1,1)$

$\alpha^j_{i 
\cdot} = \alpha_{i \ae} g^{\ae j} = A \cdot \Gamma^{-1}$

$\alpha^{ij} = \alpha_{km}g^{ki}g^{km} = (\Gamma^{-1}A \Gamma^{-1})$



\textbf{Задача 3.}

Прекрасное решение от ЕА есть в беседе