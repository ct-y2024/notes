\documentclass{article}
\usepackage[normalem]{ulem}
\usepackage[12pt]{extsizes}
\usepackage[utf8]{inputenc}
\usepackage[T2A]{fontenc}
\usepackage{amsmath}
\usepackage{amssymb}
\usepackage{hyperref}
\usepackage{amsfonts}
\usepackage{cmap}
\usepackage{multicol}
\usepackage{comment}
\usepackage{listings}
\usepackage{color}
\usepackage{colortbl}
\definecolor{bkgreen}{rgb}{0.0, 0.26, 0.15}
\usepackage[parfill]{parskip}

\usepackage{xcolor}
\usepackage[left=1.5cm,right=2cm,top=2cm,bottom=2cm,bindingoffset=0.1cm]{geometry}
\usepackage[russian]{babel}
\usepackage[pdf]{graphviz}
\usepackage{tikz}
\usepackage{etoolbox} % <--- added


\DeclareRobustCommand{\divby}{%
\mathrel{\text{\vbox{\baselineskip.65ex\lineskiplimit0pt\hbox{.}\hbox{.}\hbox{.}}}}%
}
\AtBeginEnvironment{enumerate}{\linespread{.84}\selectfont}
\newcommand{\ctd}{\begin{flushright} $\square$ \end{flushright}}

\usepackage{amsthm}
\usepackage{mathtools}
\usepackage{textcomp}
\usepackage{tikz}

%\theoremstyle{definition} % жирный заголовок, плоский текст
\newtheorem{Thm}{\underline{Теорема}} % нумерация будет "<номер subsection>.<номер теоремы>"
\newtheorem{Lm}[Thm]{\underline{Лемма}} % Нумерация такая же, как и у теорем
\newtheorem{Ex}[Thm]{Упражнение} % Нумерация такая же, как и у теорем
\newtheorem{Task}[Thm]{Задача} % Нумерация такая же, как и у теорем
\newtheorem{Example}{Пример}[section] % Нумерация такая же, как и у теорем
\newtheorem{Code}[Thm]{Код} % Нумерация такая же, как и у теорем
%\theoremstyle{plain} % жирный заголовок, курсивный текст
\newtheorem{Def}{Определение} % Нумерация такая же, как и у теорем

\newtheorem{Cons}[Thm]{Следствие} % Нумерация такая же, как и у теорем
\newtheorem{Conj}[Thm]{Гипотеза} % Нумерация такая же, как и у теорем
\newtheorem{Prop}[Thm]{Утверждение} % Нумерация такая же, как и у теорем
\newtheorem{Rem}{Замечание} % Нумерация такая же, как и у теорем
\newtheorem{Remark}[Thm]{Замечание} % Нумерация такая же, как и у теорем
\newtheorem{Img}[Thm]{Иллюстрация} % Нумерация такая же, как и у теорем

\newcommand{\deff}[1]{\underline{\textbf{#1}}}
\newcommand{\thmm}[1]{\underline{\textbf{#1}}}
\newcommand*\xor{\mathbin{\oplus}}
\newcommand{\mytilde}{\raisebox{0.5ex}{\texttildelow}}

\makeatletter
\renewcommand*\env@matrix[1][*\c@MaxMatrixCols c]{%
  \hskip -\arraycolsep
  \let\@ifnextchar\new@ifnextchar
  \array{#1}}
\makeatother

\hypersetup{
    colorlinks=true,
    linkcolor=black,
    filecolor=magenta,      
    urlcolor=blue,
    pdftitle={Linear Algebra semestr 2},
    pdfpagemode=FullScreen,
    }

\lstset{ %
  language=C++, % the language of the code
  basicstyle=\footnotesize\ttfamily, % the size of the fonts that are used for the code
  numbers=left, % where to put the line-numbers
  numberstyle=\footnotesize\color{black},  % the style that is used for the line-numbers
  stepnumber=0, % the step between two line-numbers. If it's 1, each line 
       % will be numbered
  numbersep=0.7em,       % how far the line-numbers are from the code
  backgroundcolor=\color{white!95!gray}, % choose the background color. You must add \usepackage{color}
  showspaces=false,      % show spaces adding particular underscores
  showstringspaces=false,% underline spaces within strings
  showtabs=false,        % show tabs within strings adding particular underscores
  frame=single, % adds a frame around the code
  rulecolor=\color{black},        % if not set, the frame-color may be changed on line-breaks within not-black text (e.g. commens (green here))
  tabsize=2,    % sets default tabsize to 2 spaces
  %captionpos=b,% sets the caption-position to bottom
  breaklines=true,       % sets automatic line breaking
  breakatwhitespace=false,        % sets if automatic breaks should only happen at whitespace
  %title=\lstname,       % show the filename of files included with \lstinputlisting;
       % also try caption instead of title
  identifierstyle=\color{black!50!green},  
  keywordstyle=\color{blue},      % keyword style
  commentstyle=\color{gray},      % comment style
  stringstyle=\color{purple},      % string literal style
  escapeinside={\%*}{*)},% if you want to add a comment within your code
  morekeywords={n,k},    % if you want to add more keywords to the set
  morecomment=[l][\color{black!50!green}]{\#}, % to color #include<cstdio> 
  morecomment=[s][\color{gray!50!black}]{/**}{*/}
}


% Разные определения
\newcommand{\abs}[1]{\left|{#1}\right|}
\newcommand{\norm}[1]{\lVert{#1}\rVert}
\newcommand{\stk}[2]{\stackrel{\eqref{#1}}{#2}}
\newcommand{\D}{\Delta}
\newcommand{\pderiv}[2]{\frac{\partial #1}{\partial #2}}
\newcommand{\appr}[1]{\xrightarrow[#1]{}}
\newcommand{\scal}[1]{\left\langle #1 \right\rangle}
\newcommand{\F}{\mathcal{F}}


% Числовые множества
\newcommand{\R}{\mathbb{R}}
\renewcommand{\C}{\mathbb{C}}
\newcommand{\N}{\mathbb{N}}
\newcommand{\Q}{\mathbb{Q}}
\newcommand{\Z}{\mathbb{Z}}


% Необходимость, достаточность
\newcommand{\nec}{{$\Rightarrow$}}
\newcommand{\suff}{{$\Leftarrow$}}


% Буквенные \Re и \Im
\DeclareMathOperator{\@custom@Re}{Re}
\DeclareMathOperator{\@custom@Im}{Im}
\renewcommand{\Re}{\@custom@Re}
\renewcommand{\Im}{\@custom@Im}

% Еще переопределения
\renewcommand{\emptyset}{\varnothing}

% Математические операторы
\DeclareMathOperator{\rank}{rank}
\DeclareMathOperator{\mes}{mes}
\DeclareMathOperator{\diam}{diam}
\DeclareMathOperator{\fix}{fix}
\DeclareMathOperator{\sgn}{sgn}
\DeclareMathOperator{\sign}{sgn}
\DeclareMathOperator{\vp}{v.p.}
\DeclareMathOperator{\Arg}{Arg}
\DeclareMathOperator{\Ln}{Ln}
\DeclareMathOperator{\Arcsin}{Arcsin}
\DeclareMathOperator{\Arccos}{Arccos}
\DeclareMathOperator{\Arctg}{Arctg}
\DeclareMathOperator{\Arcctg}{Arcctg}
\DeclareMathOperator{\Arsh}{Arsh}
\DeclareMathOperator{\Arch}{Arch}
\DeclareMathOperator{\Arth}{Arth}
\DeclareMathOperator{\Arcth}{Arcth}

\usepackage{dsfont}
\newcommand{\zero}{\mathds{O}}
\renewcommand{\ker}{\mathcal{K}er}
\newcommand{\rg}{rg}
\renewcommand{\span}{span}

% Интегралы до бесконечности
\newcommand{\iinf}[1]{\int\limits_{#1}^{+\infty}}
\newcommand{\ioinf}{\int\limits_{0}^{+\infty}}
\newcommand{\ipminf}{\int\limits_{-\infty}^{+\infty}}

\newcommand{\Sim}{\text{Sim }}
\newcommand{\Alt}{\text{Alt }}

\newenvironment{Word}[2]{
    \vspace{2pt}
    \textbf{words:}
    \vspace{-7pt}
    \begin{multicols}{#1}
    \begin{enumerate}[#2]       
}{\end{enumerate}\end{multicols}}

\usepackage{lipsum} % sample text
\usepackage{wrapfig}
\usepackage{minted}
\setminted{
    linenos=true,
    frame=leftline,
    fontsize=\ttfamily\small,
    framesep=4mm,
    numbersep=4pt,
    tabsize=4,
    breaklines=true,
    breakautoindent=true
}

\usepackage{subfiles}
\usepackage{fancyhdr}
\pagestyle{fancy}
\fancyhf{}

\fancyhead[C]{Линейная Алгебра} % Центральный заголовок
\fancyhead[L]{КТ ИТМО - 2 Семестр}
\fancyhead[R]{Кучерук Екатерина}


\title{Разбор КР по Тензорам.}
\author{Чепелин Вячеслав}
\date{}

\begin{document}
\maketitle
\tableofcontents
\newpage

\section{Разбор Кр прошлых лет}

\subsection{Задание 1.}

$M_2$ пространство матриц $2\times 2$. $C= \begin{pmatrix}
    3 & 5\\
    -2 & 1
\end{pmatrix}, \forall x \in M_2: f(x) = tr (XC)$

\begin{enumerate}
    \item Докажите, что $f\in M_2^*$
    \item Найдите координаты в базисе сопряженном базису $\begin{pmatrix}
        4 & 2 \\
        -1 & -6
    \end{pmatrix}, \begin{pmatrix}
        1 & 2 \\
        1 & 1 \\
    \end{pmatrix}, \begin{pmatrix}
        2 & 3\\
        1 & 0 \\
    \end{pmatrix}, \begin{pmatrix}
        3 & 1 \\
        1 &  -2
    \end{pmatrix}$
\end{enumerate}

\textbf{Решение:}

1) Пусть $X = \begin{pmatrix}
    x_{1} & x_{2}\\
    x_{3} & x_{4}
\end{pmatrix}$. Тогда $XC = \begin{pmatrix}
    3x_1 -2x_2 & * \\
    * & 5x_3 + x_4
\end{pmatrix}$. 

Тогда след равен $3x_1-2x_2 +5x_3 + x_4$. Заметим, что наше $f$ линейно, откуда $f \in M_2^*$. 

2) Теперь представим каждую матрицу, как столбики в каноническом базиск $\begin{pmatrix}
    4 \\
    2 \\
    -1\\
    -6\\
\end{pmatrix}, \begin{pmatrix}
    1 \\
    2\\
    1\\
    1\\
\end{pmatrix},\begin{pmatrix}
    2\\
    3\\
    1\\
    0
\end{pmatrix}, \begin{pmatrix}
    3 \\
    1\\
    1\\
    -2
\end{pmatrix}$. Тогда матрица перехода из канонического в наш будет:

$T = \begin{pmatrix}
    4 & 1 & 2 &3 \\
    2 & 2 &3 & 1\\
    -1 &1 & 1 &1\\
    -6 & 1 & 0 &-2
\end{pmatrix}$.  Как мы знаем $a'=aT$. Откуда найдем $a = (3,-2,5,1)$. 

Умножим и получим: $$a' = \begin{pmatrix}
    3 & -2 & 5 & 1
\end{pmatrix} \cdot \begin{pmatrix}
    4 & 1 & 2 &3 \\
    2 & 2 &3 & 1\\
    -1 &1 & 1 &1\\
    -6 & 1 & 0 &-2
\end{pmatrix} = \begin{pmatrix}
    -3 & 5 &5& 10
\end{pmatrix} $$
Либо я мог просто напросто подставить эти вектора в мою формулу!
\newpage
\subsection{Задание 2.}

\begin{enumerate}
    \item $\alpha \in T(1,2), \beta \in T(1,0)$. Найти тип и матрицу тензора $\alpha\otimes \beta$, $\alpha =\begin{pmatrix}[cc|cc]
        -1 & 1 & 2 & 0\\
        1 & 1 & -3 & 1\\
    \end{pmatrix}, \begin{pmatrix}
        2 & -1
    \end{pmatrix}$ 
    \item $\gamma \in T(1,3)$. Применить $\gamma^{ijk}_j$ и $\gamma^{i[jk]}_l$ $\gamma = \begin{pmatrix}[cc|cc]
        3 &1 & 0 & 4\\
        -2 & 2 & -3 & 0\\
        \hline 2& 2 & 4 & 0\\
        -1 & 1 & -3 & 1
    \end{pmatrix}$
\end{enumerate}

\textbf{Решение:}
\begin{enumerate}
    \item Мы получим $\gamma = \alpha \otimes \beta  \in T(2,2)$. При этом
    $\gamma^{ij}_{kl} = \alpha^{ij}_k\cdot \beta_l$. Тогда матрица будет вот такой:
    $$\gamma = \begin{pmatrix}[cc|cc]
        -2 & 2 & 1 & -1\\
        2 & 2  & -1 & -1\\
        \hline 4 &0 & -2 & 0\\
        -6 &2&3&-1
    \end{pmatrix}$$
    \item Делаем свертку: $\beta^{ik} = \begin{pmatrix}
        7 & 6 \\
        -2 & 0
    \end{pmatrix}$. 
    
    Делаем альтернирование  по 2-ум индексам. Выпишем $i=1,k=1$ $\Alt \begin{pmatrix}
        -2 & 4 \\
        2 & 0
    \end{pmatrix} = \begin{pmatrix}
        0 & 1\\
        -1 & 0
    \end{pmatrix} $. Аналогично для остальных получим:
    $\beta_2 =\begin{pmatrix}
        0 & -1& 0&\frac{1}{2}\\
        0 & 4& 0&-2\\
        1& 0& -\frac{1}{2}&0\\
        -4 & 0& 2&0\\
    \end{pmatrix}$
\end{enumerate}

\newpage
\subsection{Задание 3.}

$\alpha = (4e_1 - e_2 + 2e_3) \otimes (e_2-e_3)\otimes e_2 + (e_1 + e_2) \otimes (-e_2 + 3e_3 )\otimes (e_1 -2e_2) $
\begin{enumerate}
    \item Найти тип тензора.
    \item Найти тензор, сделанный перестанвокой $\sigma  = (kij)$
    \item Найти $\beta(\eta^1,\eta^2 ,\eta^3) $, если $\eta_1 = w^1 - w^2 + w^3; \eta^2 = w^1 + 2w^2 + w^2; \eta^3 = w^2 -2w^3$.
\end{enumerate}

\textbf{Решение:}

\begin{enumerate}
    \item тензор типа (0,3).
    \item Найдем сначала матрицу нашего тензора:
    $\alpha = \begin{pmatrix}[ccc|ccc|ccc]
        0 & -1& 3   & 0 & 6 & -10 & 0 & 0 & 0\\
        0 & -1& 3 & 0 & 1 & -5 &  0 & 0 & 0\\
        0 & 0 & 0 & 0 & 2 & -2 & 0 &  0 & 0
    \end{pmatrix}$

    Будем делать транспонрование по частям $(312) \rightarrow (132) \rightarrow (123)$

    У нас фиксирован слой. Сделаем перестановку
    Cделаем $\beta$:
    $$\beta' = \begin{pmatrix}[ccc|ccc|ccc]
        0 & 0& 0   & 0 & 0 & 0 & 0 & 0 & 0\\
        -1 & -1& 0 & 6 & 1 & 2 &  0 & 0 & 0\\
        3 & 3 & 0 & -10 & -5 & -2 & 0 &  0 & 0
    \end{pmatrix}$$
    Теперь у нас зафиксирована строка, сделаем перестановку:
    $$\beta = \begin{pmatrix}[ccc|ccc|ccc]
        0 & 0& 0   & 0 & 0 & 0 & 0 & 0 & 0\\
        -1 & 6& 0 & -1 & 1 & 0 &  0 & 2 & 0\\
        3 & -10 & 0 & 3 & -5 & 0 & 0 &  -2 & 0
    \end{pmatrix}$$

    \item $\beta (\xi_1,\xi_2,\xi_3) = \alpha (\xi_3,\xi_1,\xi_2)$. Дорешайте сами
\end{enumerate}

\subsection{Задание 4.}

Даны 3 ковектора $f^1 = \begin{pmatrix}
    1 & -1 & 1 &1
\end{pmatrix}, f^2 = \begin{pmatrix}
    2 & 0 & 3 & 0 
\end{pmatrix}, f^3 = \begin{pmatrix}
    1 & -3 & -3 0
\end{pmatrix}$
\begin{enumerate}
    \item найти существенные коордианаты 3-формы $f =f^1 \wedge f^2 \wedge f^3$
    \item выписать матицу в пространстве $f$
    \item найти $f(\xi_1,\xi_2,\xi_3)$
\end{enumerate}
\textbf{Решение}

Буквально номер из дзшки. Смотрите разборы практик и дз.


\newpage
\section{Информация о курсе}

Поток — y2024.\newline
Группы M3138-M3139.\newline
Преподаватель --- Кучерук Екатерина Аркадьевна.\par

Данный разбор сделан не в коммерческих целях, я не хочу никого обидеть, я просто пишу конспекты для себя плак плак плак
\begin{center}
   \includegraphics[height=17cm]{assets/cute-girl.jpg}
\end{center}

\end{document}
