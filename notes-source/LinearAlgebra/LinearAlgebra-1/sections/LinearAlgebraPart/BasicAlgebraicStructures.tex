\section{Основные алгебраические структуры}
\subsection{Операции, группа, кольцо, поле}
\subsubsection{Законы композиции.}

$f: A \times B \rightarrow C$ - функция, отображение.

$\forall (a,b): a \in A, b \in B: \exists! c \in C$ --- \deff{закон внешней композиции.}

$f: A \times A \rightarrow A$ --- \deff{закон внутренней композиции} или алгебраическая операция, бинарная операция.

\subsubsection{Ассоциативность, коммутативность алгебраических операций.}

Возьмем операцию $*: A \times A \rightarrow A$:

$a * b = b * a$ --- \deff{коммутативность}.

$a * (b *c) = (a * b) *c$ --- \deff{ассоциативность}.


\subsubsection{Алгебраическая структура, группа, кольцо, поле. Свойства.}

\deff{Алгебраическая структура} --- множество с набором $\Omega$ --- операция и отношений на ней, с некоторой системой аксиом. Обозначают $(A, \Omega)$

\deff{Группа} (A, \{*\}):

* --- групповая операция, чаще всего обозначается как ``$\cdot$'' --- умножение, или как ``$+$'' --- сложение.

``$\cdot$'' --- мультипликативная запись, где $e$ --- единица, а $-a$ --- обратный.

``$+$'' --- аддитивная запись, где $e$ заменяется на $0$ --- нулевой, а $-a$ --- противоположный.

\begin{enumerate}
    \item $a * (b * c) = (a * b) * c$ --- ассоциативность.
    \item $\exists e:\forall a: a * e = e * a = a$ --- существование нейтрального элемента $e$.
    \item $\forall a: \exists (-a): a+ (-a) = e$ --- существование обратного элемента $(-a)$.
\end{enumerate}

Если группа обладает еще и коммутативностью, то такая группа --- \deff{абелева}:

\begin{enumerate}[resume]
    \item $a * b = b * a$
\end{enumerate}

\deff{Кольцо} (A, \{+, $\cdot$\}):

\begin{enumerate}
    \item[1--4.] Абелева групппа по сложению.
    \setcounter{enumi}{4}
    \item $a \cdot (b+c) = a\cdot b + a\cdot c$ --- левая дистрибутивность.
    \item $(b+c)\cdot a  = b \cdot a + c \cdot a$  --- правая дистрибутивность.
\end{enumerate}

\deff{Поле}  (A, \{+, $\cdot$\}):
\begin{enumerate}
    \item[1--5.] Кольцо по сложению.
    \item[6--9.] Абелева группа по умножению для ненулевых элементов.
\end{enumerate}

Поле --- это ассоциативное коммутативное кольцо с единицей, то есть для $\forall$ ненулевого элемента $\exists$ обратный, а вот у нуля обратного нет и это нормально.

Свойства кольца:

\begin{enumerate}
    \item $0 \cdot a = 0$
    \item $a+x = a+ y \rightarrow x=y$
    \item $a + x = b$ имеет единственное решение $x = -a + b$
    \item $0$ --- единственен.
    \item $1$ --- единственна в кольце с единицей.
\end{enumerate}

\subsection{Линейное пространство, алгебра, свойства.}

$K - $ поле, $V$ - множество. $+: V \times V \rightarrow V$, $\cdot: K\times V \rightarrow V$. Если все, что сказано ниже выполнено  $\forall \phi, \lambda \in K, a,b \in V$.

\begin{enumerate}
    \item[1--4.] Абелева группа по сложению.
    \setcounter{enumi}{4}
    \item $ \phi(\lambda(a)) = \lambda(\phi(a))$.
    \item $\lambda (a+b) = \lambda a + \lambda b$.
    \item $a ( \phi + \lambda) = a\phi + a \lambda $.
    \item $\exists 1: a \cdot 1 =a$.
\end{enumerate}

То тогда такую систему называют \deff{линейным пространством} над полем $K$.

Если добавить еще одну операцию $\times: V \times V \rightarrow V$.

\begin{enumerate}[resume]
    \item $(a+b)\times c = a \times c + b \times c $

          $c\times (a+b) = c \times a + c \times b$

    \item $\lambda (a \times b) = (\lambda a )\times b = a \times (\lambda b)$
\end{enumerate}

То такую структуру называют \deff{алгеброй}.

\begin{enumerate}[resume]
    \item добавим коммутативность $\times$ --- коммутативная алгебра.

    \item добавим ассоциативность $\times$ --- ассоциативная алгебра.

    \item добавим единицу --- унитальная алгебра.

    \item добавим обратное (для ненулевых элементов) --- алгебра с делением.
\end{enumerate}

\subsection{Нормированные линейные пространства и алгебры.}

\deff{Нормированное пространство} --- линейное пространство над $\mathbb{R}(\mathbb{C})$ с нормой.

\deff{Норма} $||\cdot||:V \rightarrow \mathbb{R}(\mathbb{C})$, удовлетворящее:

\begin{enumerate}
    \item $\forall x, y \in V : ||x|| + ||y||\geq ||x+y||$.
    \item $\forall x \in V : ||x||\geq 0$, причем $||x||=0 \Leftrightarrow$ $x = 0$.
    \item $\forall x \in V : \forall \alpha \in \mathbb{R}(\mathbb{C}) : ||\alpha x|| = |\alpha|||x||$.
\end{enumerate}

Алгебра называется \deff{нормированной}, если существует норма согласованная с умножением:

$||ab||\leq ||a|| \cdot ||b||$.

\subsection{Отношение эквивалентности, фактор-структуры.}



\deff{Бинарное отношение} $\sim$ на множестве $X$ --- \deff{отношение эквивалентности}, если оно
\begin{itemize}
    \item Рефлексивно: $\forall x\in X~x\sim x$.
    \item Симметрично: $\forall x,y\in X~x\sim y\leftrightarrow y\sim x$.
    \item Транзитивно: $\forall x,y,z\in X~x\sim y\land y\sim z\rightarrow x\sim z$.
\end{itemize}
Если $\sim$ --- бинарное отношение на $X$, то множества $M_a=\{x\in X\mid x\sim a\}$ называются классами эквивалентности , а множество $X/\sim=\{M_a\mid a\in X\}$ --- \deff{фактормножеством} (или {факторпространством}) $X$ по $\sim$.

\thmm{Свойства классов эквивалентности.}
\begin{enumerate}
    \item $\forall a\in X~M_a\neq\varnothing$.
    \item $\forall a,b\in X$ выполнено либо $M_a=M_b$, либо $M_a\cap M_b=\varnothing$.
    \item $\bigcup\limits_{a\in X}M_a=X$.
\end{enumerate}

Если у нас есть множество $X$, а $M$ --- какое-то множество, состоящее из непустых взаимно непересекающихся подмножеств $X$, в объединении дающих $X$. Тогда $M$ называется \deff{разбиением} $X$.


Любое разбиение $X$ является факторпространством $X$ по некоторому отношению эквивалентности. Доказательство этого тривиально, если вы представите отношения как ребра в графе, а классы эквивалентности - компоненты


