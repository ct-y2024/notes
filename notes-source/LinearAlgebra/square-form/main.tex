\documentclass{article}
\usepackage[normalem]{ulem}
\usepackage[12pt]{extsizes}
\usepackage[utf8]{inputenc}
\usepackage[T2A]{fontenc}
\usepackage{amsmath}
\usepackage{amssymb}
\usepackage{hyperref}
\usepackage{amsfonts}
\usepackage{cmap}
\usepackage{multicol}
\usepackage{comment}
\usepackage{listings}
\usepackage{color}
\usepackage{colortbl}
\definecolor{bkgreen}{rgb}{0.0, 0.26, 0.15}
\usepackage[parfill]{parskip}

\usepackage{xcolor}
\usepackage[left=1.5cm,right=2cm,top=2cm,bottom=2cm,bindingoffset=0.1cm]{geometry}
\usepackage[russian]{babel}
\usepackage[pdf]{graphviz}
\usepackage{tikz}
\usepackage{etoolbox} % <--- added


\DeclareRobustCommand{\divby}{%
\mathrel{\text{\vbox{\baselineskip.65ex\lineskiplimit0pt\hbox{.}\hbox{.}\hbox{.}}}}%
}
\AtBeginEnvironment{enumerate}{\linespread{.84}\selectfont}
\newcommand{\ctd}{\begin{flushright} $\square$ \end{flushright}}

\usepackage{amsthm}
\usepackage{mathtools}
\usepackage{textcomp}
\usepackage{tikz}

%\theoremstyle{definition} % жирный заголовок, плоский текст
\newtheorem{Thm}{\underline{Теорема}} % нумерация будет "<номер subsection>.<номер теоремы>"
\newtheorem{Lm}[Thm]{\underline{Лемма}} % Нумерация такая же, как и у теорем
\newtheorem{Ex}[Thm]{Упражнение} % Нумерация такая же, как и у теорем
\newtheorem{Task}[Thm]{Задача} % Нумерация такая же, как и у теорем
\newtheorem{Example}{Пример}[section] % Нумерация такая же, как и у теорем
\newtheorem{Code}[Thm]{Код} % Нумерация такая же, как и у теорем
%\theoremstyle{plain} % жирный заголовок, курсивный текст
\newtheorem{Def}{Определение} % Нумерация такая же, как и у теорем

\newtheorem{Cons}[Thm]{Следствие} % Нумерация такая же, как и у теорем
\newtheorem{Conj}[Thm]{Гипотеза} % Нумерация такая же, как и у теорем
\newtheorem{Prop}[Thm]{Утверждение} % Нумерация такая же, как и у теорем
\newtheorem{Rem}{Замечание} % Нумерация такая же, как и у теорем
\newtheorem{Remark}[Thm]{Замечание} % Нумерация такая же, как и у теорем
\newtheorem{Img}[Thm]{Иллюстрация} % Нумерация такая же, как и у теорем

\newcommand{\deff}[1]{\underline{\textbf{#1}}}
\newcommand{\thmm}[1]{\underline{\textbf{#1}}}
\newcommand*\xor{\mathbin{\oplus}}
\newcommand{\mytilde}{\raisebox{0.5ex}{\texttildelow}}

\makeatletter
\renewcommand*\env@matrix[1][*\c@MaxMatrixCols c]{%
  \hskip -\arraycolsep
  \let\@ifnextchar\new@ifnextchar
  \array{#1}}
\makeatother

\hypersetup{
    colorlinks=true,
    linkcolor=black,
    filecolor=magenta,      
    urlcolor=blue,
    pdftitle={Linear Algebra semestr 2},
    pdfpagemode=FullScreen,
    }

\lstset{ %
  language=C++, % the language of the code
  basicstyle=\footnotesize\ttfamily, % the size of the fonts that are used for the code
  numbers=left, % where to put the line-numbers
  numberstyle=\footnotesize\color{black},  % the style that is used for the line-numbers
  stepnumber=0, % the step between two line-numbers. If it's 1, each line 
       % will be numbered
  numbersep=0.7em,       % how far the line-numbers are from the code
  backgroundcolor=\color{white!95!gray}, % choose the background color. You must add \usepackage{color}
  showspaces=false,      % show spaces adding particular underscores
  showstringspaces=false,% underline spaces within strings
  showtabs=false,        % show tabs within strings adding particular underscores
  frame=single, % adds a frame around the code
  rulecolor=\color{black},        % if not set, the frame-color may be changed on line-breaks within not-black text (e.g. commens (green here))
  tabsize=2,    % sets default tabsize to 2 spaces
  %captionpos=b,% sets the caption-position to bottom
  breaklines=true,       % sets automatic line breaking
  breakatwhitespace=false,        % sets if automatic breaks should only happen at whitespace
  %title=\lstname,       % show the filename of files included with \lstinputlisting;
       % also try caption instead of title
  identifierstyle=\color{black!50!green},  
  keywordstyle=\color{blue},      % keyword style
  commentstyle=\color{gray},      % comment style
  stringstyle=\color{purple},      % string literal style
  escapeinside={\%*}{*)},% if you want to add a comment within your code
  morekeywords={n,k},    % if you want to add more keywords to the set
  morecomment=[l][\color{black!50!green}]{\#}, % to color #include<cstdio> 
  morecomment=[s][\color{gray!50!black}]{/**}{*/}
}


% Разные определения
\newcommand{\abs}[1]{\left|{#1}\right|}
\newcommand{\norm}[1]{\lVert{#1}\rVert}
\newcommand{\stk}[2]{\stackrel{\eqref{#1}}{#2}}
\newcommand{\D}{\Delta}
\newcommand{\pderiv}[2]{\frac{\partial #1}{\partial #2}}
\newcommand{\appr}[1]{\xrightarrow[#1]{}}
\newcommand{\scal}[1]{\left\langle #1 \right\rangle}
\newcommand{\F}{\mathcal{F}}


% Числовые множества
\newcommand{\R}{\mathbb{R}}
\renewcommand{\C}{\mathbb{C}}
\newcommand{\N}{\mathbb{N}}
\newcommand{\Q}{\mathbb{Q}}
\newcommand{\Z}{\mathbb{Z}}


% Необходимость, достаточность
\newcommand{\nec}{{$\Rightarrow$}}
\newcommand{\suff}{{$\Leftarrow$}}


% Буквенные \Re и \Im
\DeclareMathOperator{\@custom@Re}{Re}
\DeclareMathOperator{\@custom@Im}{Im}
\renewcommand{\Re}{\@custom@Re}
\renewcommand{\Im}{\@custom@Im}

% Еще переопределения
\renewcommand{\emptyset}{\varnothing}

% Математические операторы
\DeclareMathOperator{\rank}{rank}
\DeclareMathOperator{\mes}{mes}
\DeclareMathOperator{\diam}{diam}
\DeclareMathOperator{\fix}{fix}
\DeclareMathOperator{\sgn}{sgn}
\DeclareMathOperator{\sign}{sgn}
\DeclareMathOperator{\vp}{v.p.}
\DeclareMathOperator{\Arg}{Arg}
\DeclareMathOperator{\Ln}{Ln}
\DeclareMathOperator{\Arcsin}{Arcsin}
\DeclareMathOperator{\Arccos}{Arccos}
\DeclareMathOperator{\Arctg}{Arctg}
\DeclareMathOperator{\Arcctg}{Arcctg}
\DeclareMathOperator{\Arsh}{Arsh}
\DeclareMathOperator{\Arch}{Arch}
\DeclareMathOperator{\Arth}{Arth}
\DeclareMathOperator{\Arcth}{Arcth}

\usepackage{dsfont}
\newcommand{\zero}{\mathds{O}}
\renewcommand{\ker}{\mathcal{K}er}
\newcommand{\rg}{rg}
\renewcommand{\span}{span}

% Интегралы до бесконечности
\newcommand{\iinf}[1]{\int\limits_{#1}^{+\infty}}
\newcommand{\ioinf}{\int\limits_{0}^{+\infty}}
\newcommand{\ipminf}{\int\limits_{-\infty}^{+\infty}}

\newcommand{\Sim}{\text{Sim }}
\newcommand{\Alt}{\text{Alt }}

\newenvironment{Word}[2]{
    \vspace{2pt}
    \textbf{words:}
    \vspace{-7pt}
    \begin{multicols}{#1}
    \begin{enumerate}[#2]       
}{\end{enumerate}\end{multicols}}

\usepackage{lipsum} % sample text
\usepackage{wrapfig}
\usepackage{minted}
\setminted{
    linenos=true,
    frame=leftline,
    fontsize=\ttfamily\small,
    framesep=4mm,
    numbersep=4pt,
    tabsize=4,
    breaklines=true,
    breakautoindent=true
}

\usepackage{subfiles}
\usepackage{fancyhdr}
\pagestyle{fancy}
\fancyhf{}

\fancyhead[C]{Линейная Алгебра} % Центральный заголовок
\fancyhead[L]{КТ ИТМО - 2 Семестр}
\fancyhead[R]{Кучерук Екатерина}


\title{Разбор КР по Квадратичным формам.}
\author{Чепелин Вячеслав}
\date{}

\begin{document}
\maketitle
\tableofcontents
\newpage

\section{Разбор Кр прошлых лет}
\subsection{Задание 1.}

Как сказала ЕА, в данном задании нужно будет воспользоваться  одним из трех приведений к канонич. виду. Оба они есть в основном конспекте. Здесь будет разобран вот такой пример:
$$f(x) = 9x_1^2 +5x_2^2+5x_3^2  +8x_4^2 + 8x_2x_3- 4x_2x_4 + 4x_3x_4$$
Нужно привести ортогональным преобразованием к каноническому виду и показать его.

\textbf{Решение:}

Напишем матрицу оператора, соответствующую кв. форме:
$$\begin{pmatrix}
    9 & 0& 0& 0\\
    0 & 5 & 4& -2\\
     0 & 4 & 5 &2 \\
      0 & -2 & 2&8 \\
\end{pmatrix}$$
Теперь найдем собственные числа и вектора нашей матрицы:
$$\det(A-t\varepsilon) =  
t^4-27t^3+243t^2-729t = t(t-9)^3$$
Откуда получаем, что собственные числа нашей матрицы это 0 и 9.

Найдем собственные вектора, решим соответсв. СЛОУ:
$$\begin{pmatrix}[cccc|c]
    9-9 & 0& 0& 0 & 0\\
    0 & 5-9 & 4& -2& 0\\
     0 & 4 & 5-9 &2 & 0\\
      0 & -2 & 2&8-9 & 0\\
\end{pmatrix}
\begin{pmatrix}[cccc|c]
    9-0 & 0& 0& 0 & 0\\
    0 & 5-0 & 4& -2& 0\\
     0 & 4 & 5-0 &2 & 0\\
      0 & -2 & 2&8-0 & 0\\
\end{pmatrix}$$
Решив их, получим, что  $V_9 = \span (\begin{pmatrix}
    1 \\
    0\\
    0\\
    0\\
\end{pmatrix}, \begin{pmatrix}
    0 \\
    1\\
    1\\
    0\\
\end{pmatrix}, \begin{pmatrix}
    0 \\
    -1\\
    0\\
    2\\
\end{pmatrix}), V_0 = \span (\begin{pmatrix}
    0 \\
    2\\
    -2\\
    1\\
\end{pmatrix})$

Теперь сделаем наши вектора ортогональными и нормированы (так как нам надо, чтобы матрица перехода, то есть $Q$ была ортогональной). 

Ортогонализуем и получим:
$V_9 = \span (\begin{pmatrix}
    1 \\
    0\\
    0\\
    0\\
\end{pmatrix}, \begin{pmatrix}
    0 \\
    1\\
    1\\
    0\\
\end{pmatrix}, \begin{pmatrix}
    0 \\
    -1\\
    1\\
    4\\
\end{pmatrix}),V_0 = \span (\begin{pmatrix}
    0 \\
    2\\
    -2\\
    1\\
\end{pmatrix})$

Теперь отнормируем и получим матрицу $Q = \begin{pmatrix}
    0 & 1 & 0 & 0\\
    \frac{2}{3} & 0 & \frac{1}{\sqrt{2}}& \frac{-1}{\sqrt{18}}\\
    \frac{-2}{3} & 0 & \frac{1}{\sqrt{2}} & \frac{1}{\sqrt{18}}\\
    \frac{1}{3}& 0 & 0 & \frac{4}{\sqrt{18}}\\
\end{pmatrix}, \Lambda = \begin{pmatrix}
    0 & 0 & 0 & 0\\
    0 & 9 & 0 & 0\\
    0 & 0 & 9 & 0\\
    0 & 0 & 0 & 9\\
\end{pmatrix}$



$B = Q^T A Q  = \Lambda$, что и требовалось найти. $rg f = 3, \sigma(f) = (3,0,1), f \geq 0 $
\newpage 

\subsection{Задание 2.}

Найти преобразование переводящее из $f$ в $g$:
$$f(x) = x_1x_2  +x_2x_3 +  x_3x_4 + x_4x_1$$
$$g(y) = y_1^2 - 5y_2^2  + y_3^2 +4y_1y_2 + 2y_1 y_3 + 4y_2y_3$$
\textbf{Решение:}

Чтобы это сделать, воспользуюсь методом Лагранжа и приведу обе форму к каноническому виду.

Сначала приведем форму $f$. 

\begin{enumerate}
    \item У нас нет квадратов, а значит их надо выделить:
    $$\begin{cases}
        x_1 = y_1 + y_2\\
        x_2  = y_1 - y_2\\
        x_3 = y_3 \\
        x_4 = y_4
    \end{cases}: f(y) = y_3 y_1 + y_4 y_1 - y_3 y_2 + y_4 y_2 + y_3 y_4 + y_1^2 - y_2^2$$
    \item Теперь не забудем, что мы сделали такое преобразование и продолжим по Лагранжу преобразовывать нашу форму:
    $$f(y) = (y_1^2 + y_3y_1 + y_4 y_1) -y_2^2 -y_3y_2+y_4y_2 + y_3y_4 =$$$$= \left(y_1 + \frac{1}{2}y_3 + \frac{1}{2}y_4\right)^2-\frac{1}{4}y_3^2 - \frac{1}{4}y_4^2 - \frac{1}{2}y_3 y_4  -y_2^2 -y_3y_2+y_4y_2 + y_3y_4 =  $$
    $$=z_1^2 +(-1)(y_2^2 +y_3y_2 - y_4y_2)  -\frac{1}{4}y_3^2  -\frac{1}{4}y_4^2 +
\frac{1}{2}y_3y_4=$$$$= z_1^2 + (-1)\left(y_2 + \frac{1}{2}y_3 - \frac{1}{2}y_4\right)^2 + \frac{1}{4}y_3^2 +\frac{1}{4}y_4^2 - \frac{1}{2}y_3y_4-\frac{1}{4}y_3^2  -\frac{1}{4}y_4^2 +
\frac{1}{2}y_3y_4 = z_1^2 - z_2^2 + 0z_3 + 0z_4$$
$$\begin{cases}
    z_1 = y_1 + \cfrac{1}{2}y_3 + \cfrac{1}{2}y_4\\
    z_2 = y_2 + \cfrac{1}{2}y_3 - \cfrac{1}{2}y_4\\
    z_3 = y_3\\
    z_4 = y_4
\end{cases} \Leftrightarrow \begin{cases}
    y_1  = z_1 - \cfrac{1}{2}z_3 - \cfrac{1}{2}z_4\\
    y_2  =z_2 - \cfrac{1}{2}z_3 + \cfrac{1}{2}z_4 \\
    y_3 = z_3 \\
    y_4 = z_4 \\
\end{cases}$$
Откуда $x = Q_1 y = \begin{pmatrix}
    1 & 1 & 0 & 0\\
     1 & -1 & 0 & 0\\
      0 & 0 & 1 & 0\\
       0 & 0 & 0 & 1\\
\end{pmatrix}y$, а $y = Q_2z = \begin{pmatrix}
    1 & 0 & \frac{-1}{2} & \frac{-1}{2}\\
       0 & 1 & \frac{-1}{2} &\frac{1}{2} \\
       0 & 0 & 1 & 0\\
       0 & 0 & 0 & 1\\
\end{pmatrix}z$
\end{enumerate}
Получаю, что мое преобразование это $Q_1Q_2$.

Теперь приведем форму $g$ к каноническому:
\begin{enumerate}
    \item $$g(y) = y_1^2 - 5y_2^2  + y_3^2 +4y_1y_2 + 2y_1 y_3 + 4y_2y_3 = (y_1^2 + 4y_1y_2 + 2y_1 y_3)- 5y_2^2  + y_3^2+4y_2y_3= $$
    $$= (y_1 + 2y_2 + y_3)^2 - 4y_2^2 -y_3^2 - 4y_2y_3- 5y_2^2  + y_3^2+4y_2y_3 = $$
    $$= z_1^2 - 9y_2^2 = z_1^2 - 1(3y_2)^2 + 0z_3 + 0z_4$$
    Откуда:
    $$\begin{cases}
        z_1 = y_1 + 2y_2 + y_3\\
        z_2 =  3y_2 \\
        z_3 = y_3 \\ z_4 = y_4
        \end{cases} \Leftrightarrow \begin{cases}
            y_1= z_1-z_3-\frac{2}{3}z_2\\
            y_2 = \frac{1}{3}z_2\\
            y_3 = z_3 \\
            y_4 = z_4\\
        \end{cases}$$

    Получаю, что $y = Q_3 z = \begin{pmatrix}
        1 &  -\frac{2}{3} & -1 & 0\\
         0 &  \frac{1}{3} & 0 & 0\\
          0 &  0 & 1 & 0\\
           0 &  0 & 0 & 1\\
    \end{pmatrix}$
\end{enumerate}

У нас совпали $\sigma$.

Откуда мое нужное преобразование из $f$ в $g$ это $Q_1Q_2Q_3^{-1}$ 
\newpage
\subsection{Задание 3.}

Найти линейное  невырожденное преобразование, которое переводит одновременно $f,g$ в канонический вид.
$$f(x) = x_1^2 + 4x_2^2 + 2x_3^2 + x_4^2 + 2x_1x_3$$
$$g(x) = 8x_1^2 -28x_2^2  + 14x_3^2 +16x_1x_2 + 14x_1x_3 +32x_2x_3$$

\textbf{Решение:}

Напишем матрицы обеих кв. форм: $A_1 = \begin{pmatrix}
     1 & 0 & 1 & 0\\
      0 & 4 & 0 & 0\\
      1& 0 & 2 & 0\\
       0 & 0 & 0 & 1\\
\end{pmatrix}$, $A_2 = \begin{pmatrix}
     8 & 8 & 7 & 0\\
      8 & -28 & 16 & 0\\
      7 & 16 & 14 & 0\\
       0 & 0 & 0 & 0\\
\end{pmatrix}$
Заметим, что $f$ у нас $\geq 0 $ по критерию Сильвестра

Приведем кв форму $f$ к каноническому виду методом Лагранжа:

\begin{enumerate}
    \item $$f(x) = x_1^2 + 4x_2^2 + 2x_3^2 + x_4^2 + 2x_1x_3 = (x_1^2  +2x_1x_3) + 4x_2^2 + 2x_3^2 + x_4^2 = (x_1+x_3)^2 + (2x_2)^2 + x_3^2 +x_4^2$$
    $$\begin{cases}
     z_1 = x_1 +x_3\\
     z_2 = 2x_2\\
     z_3 = x_3\\
     z_4 = x_4\\
    \end{cases} \Leftrightarrow \begin{cases}
        x_1 = z_1 -z_3\\
        x_2 = \frac{z_2}{2}\\
        x_3 = z_3\\
        x_4 = z_4
    \end{cases}$$
\end{enumerate}

$x= Qz = \begin{pmatrix}
    1 & 0& -1 & 0\\
    0 & \frac{1}{2} & 0& 0\\
    0 & 0 & 1& 0\\
    0 & 0 & 0& 1\\
\end{pmatrix}$

Теперь найдем куда перейдет $g$ при нашем преобразовании:
$$B = Q_1^TA_2 Q_1 = \begin{pmatrix}
 8	& 4	&-1	&0\\
 4	&-7	& 4	&0\\
-1	& 4	 &8	&0\\
 0&	 0	& 0&	0
\end{pmatrix}$$

Теперь найдем ортогональное преобразование для нашего нового вида $g$.

$\lambda_1 = 0, \lambda_2 = 9, \lambda_3 = -9$. Возьмем собственные вектора:
$$V_0 = \span(\begin{pmatrix}
    0\\
    0\\
    0\\
    1\\
\end{pmatrix}), V_9 =\span (\begin{pmatrix}
    2 \\
    1 \\
    2\\
    0
\end{pmatrix}, \begin{pmatrix}
    -1 \\
    0 \\
    1\\
    0
\end{pmatrix}), V_{-9} = \span(\begin{pmatrix}
    1 \\
    -4\\
    1\\
    0
\end{pmatrix})$$
А дальше аналогично первому номеру 1 и получаем $Q_2$. Ответом будет матрица $Q_1 Q_2$

\newpage
\section{Информация о курсе}

Поток — y2024.\newline
Группы M3138-M3139.\newline
Преподаватель --- Кучерук Екатерина Аркадьевна.\par

Данный разбор сделан не в коммерческих целях, я не хочу никого обидеть, я просто пишу конспекты для себя плак плак плак
\begin{center}
   \includegraphics[height=17cm]{assets/linear_algebra.jpg}
\end{center}

\end{document}
