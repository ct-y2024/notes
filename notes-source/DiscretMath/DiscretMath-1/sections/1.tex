
\subsection{Введение определний}
Множество - неопределенное понятие. A - какое-то множество. Мы умеем понимать:

$x \in A$ или $x \notin A$ 

В множестве несколько одинаковых элементов быть не может(вопрос бессмысленен:  $\forall x, \forall A: x \in  A  $ или $ x \notin A).$

Будем считать, что есть $U$ ("универсум") и все множества в нем лежат.

Мы можем спокойно работать с множеством натуральных, целых, рациональных, вещественных, иногда комплексных. Обычно будет понятно из контекста какой у нас универсум.

\textbf{Примеры задания множества}

$\mathbb{B}$ = \{0,1\}

$C:=\{x \in A: P(x) = 1\}$, где P(x) - булевая функция (предикат).

C множеством можно делать много операций:

\textbf{1) Объединение}

$A \cup B = \{x: x \in A$  или $x \in B \}$

\textbf{2) Пересечние}

$A \cap B = \{x: x \in A$ и $x \in B \}$

\textbf{3) Вычитание}

$A \textbackslash B = \{x: x \in A$ и $x \notin B \}$

\textbf{4) Исключающее объединение (xor)}

$A \oplus B = (A \textbackslash B) \cup (B \textbackslash A)$

\textbf{5) Отрицание}

$\bar{A} = U \textbackslash A$

\textbf{Произведение множеств:}

Декартово(прямое) произведение X и Y обозначается $X \times Y$, на языке кванторов:

$X \times Y = \{(x,y):x \in X, y \in Y\}$ 

$A \times A = A^2$

Нет доказательства того, что множество пар существует. В теории множеств это закреплено аксиоматически.

$(A \times B) \times C = A \times (B \times C)$ --- пренебрегаем проблемой несоответсвия типов:
Вместо $\{a, \{b, c\}\}$ и $\{\{a, b\}, c\}$ считаем их равными $\{a, b, c\}$
На языке С++ вместо $pair(a, pair(b, c))$  и $pair(pair(a, b), c)$  воспринимаем элементы объединения как $tuple(a, b, c)$

\textbf{Произведение семейства множеств}

Пусть есть множество значений $B = \{a, b, c\}$  и множество индексов $A = \{1, 2, 3\}$. Рассмотрим произведение семейства такого, что $\forall \alpha, \beta \in A: B_\alpha = B_\beta = B$ 

$\prod_{\alpha\in A}{B_\alpha}$ - множество всех способов выбрать одно из возможных значений B для каждого индекса A

Прим. лек. Элемент такого множества похож на map в C++: В качестве ключа выступает индекс из $A$, в качестве значения - элемент из $B$. Произведение выше - множество всех возможных map, которые можно составить при данных $A$ и $B$.

\textbf{Дизъюнктное объединение}

Непересекающиеся множества называют дизъюнктными. Существует специальный знак объединения дизъюнктных множеств: $\sqcup$ 

$A \sqcup B$

Знак дизъюнктного объединения используется как синтаксический сахар, чтобы указать, что множества не пересекаются, или напомнить об этом.

\textbf{Функция} 

$f:A \rightarrow B$

$X \subset Y$ --- "мн-во X содержится в Y"; равносильно $x \in X \rightarrow x \in Y$

$f \subset A \times B$. Выполнено $\forall x\ \in A \exists!\,y \in B: (x,y) \in f$

$B^A = \{f:A \rightarrow B\}$ - множество всех функций перехода из $A$ в $B$

$\{0, 1\}^{A}=2^{A}=\mathbb{B}^{A}$ -каждому элементу сопоставили 1 или 0. Удобно брать подмножества.


\textbf{Инъекция.} Если $x \neq y$, то $f(x) \neq f(y)$

\textbf{Сюръекция.} $\forall y \in B: \exists x: f(x)=y$

\textbf{Биекция} = Инъекция + Сюръекция = Взаимнооднозначное соотвествие

\textbf{Частичная функция}. 

$f: A_1 \subset A \rightarrow B$ - Переводит в $B$ только часть множества $A$

\textbf{Другие обозначения}

$\varnothing$ - пустое множество

$A^0 = \{<>\}$ - пустой картеж = void = множество с ничего внутри

$\varnothing \times A = \varnothing$

$1 \times A = A$ - зачем-то приклеили  каждому элементу ничего

\subsection*{Отношения}

(бинарные) отношения между A и B

$R \subset A \times B$ - т.е. отношение R задает граф переходов из $A$ в $B$

Прим. Функция - частный случай отношения. Не любое отношение является функцией, т.к. из одного аргумента отношение может вернуть множество значений.

\textbf{Пример 1.}

$A,B = \mathds{N}$

$\le \{(a,b): a<b\}$ Обозначается aRb (инфиксная запись). Здесь мы задали критерий принадлежности пары к отношению

$<$ - такое отношение, что $a$ должно быть меньше $b$.

\textbf{Пример 2.}

$\mathds{N} \times 2^\mathds{N}$

$R = \in $. $(17, \{1, 2, 17, 256\}) \in R$. $(2, \{x | x - \text{нечетный}\}) \notin R$

Пару $(\mathds{N}, \mathds{N})$ рассмотреть нельзя, так как она не принадлежит нашему универсуму.



$R \subset A \times A$ - отношение на A.

Отношения можно изображать в виде ориентированных графов.
Виды отношений: 

1) \textbf{Рефлексивность}.

$\forall x \in A: aRa$, то есть (a,a) входит в наше отношение.

2) \textbf{Антирефлексивность}.

$\forall x \in A: a\textbackslash Ra$, то есть (a,a) не входит в наше отношение.

3) \textbf{Симметричность}.

Если $aRb$, то $bRa$.

4) \textbf{Антисимметричность} 

Если $a\neq b$ и $aRb$, то $a\textbackslash Rb$.

Эквивалентно $aRb, bRa \rightarrow a=b$

5) \textbf{Транзитивность}.

Если $aRb$ и $bRc$, то $aRc$.

\textbf{Классы отношений.}

Отношение частичного порядка - 1,4,5 (строгий 2,4,5). Представление на графе будет выглядеть как направленный ациклический граф, но у каждой вершины есть петля

Отношение линейнего порядка - частичный и $\forall a,b: aRb$ или $bRa$.

Отношение полного порядка - линейный и $\forall $множество X имеет min элемент.

Отношение эвивалентности - 1,3,5.

$X,R$ - эквивалентна на $X$, тогда $A\in X /R$

\textbf{Классы эквивалентности}

Отношение R на множестве X может разбиться на классы эквивалентности такие, что если $a, b$ в одном классе, то  $aRb$, если $a, b$ в разных классах, то $a\textbackslash Rb$. Если представить на графе эквивалентное отношение R(наличие ребра), то все компоненты будут полными графами с петлями, а разные компоненты будут принадлежать разным классам эквивалентности.
