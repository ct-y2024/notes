
\subsection{Базисы}

\textbf{Базис} - набор булевых функций.

\textit{F} называется \textbf{полным базисом}, если используя только булевые функций из \textit{F} мы можем задать любую булеву функцию. Пример:

$\{\wedge, \vee, \neg\}$ ---  стандартный полный базис. Можем выкинуть $\wedge$ либо $\vee$ 
 и он останется полным базисом.

\textbf{Теорема.} Пусть $F$ - полный базис. $G$ - множество функций.
$\forall f \in F:$ можно задать формулой над $G$.
Тогда G - полный базис.

\begin{enumerate}

\item[] \uline{Доказательство.}

Так как $F$ - базис, через него можно выразить любую булеву функцию. Тогда давайте рассмотрим дерево операций над $F$ для какой-то булевой функции и докажем, что для этой булевой функции мы можем построить дерево операций над $G$. Возьмем дерево операций над $F$ и рассмотрим его конкретную вершину. Это какая-то функция $f$, про которую мы знаем, что $f \in F$. Раз так, то мы можем заменить ее на ее формулу над $G$. Сделаем так со всеми вершинами в дереве операций над $F$ и получим дерево операций над $G$. Раз над $G$ можно задать любую функцию, то $G$ - базис. Q.E.D.

\end{enumerate}

Так мы можем доказать, что $\{\vee, \neg\}$, $\{\wedge, \neg\}$ - полные базисы.

\subsection{Полином Жегалкина}

$\{\wedge, \oplus, 0\}$ --- \textbf{арифметический базис.}

Выражение булевой функции в арифметическом базисе --- \textbf{полином Жегалкина}.

Приведенный полином Жегалкина --- вид полинома Жегалкина, в котором опущены все произведения с 0. 

\textbf{Теорема.}  $\forall$ функции $\exists!$ приведенный полином Жегалкина
\begin{enumerate}
\item[] \uline{Доказательство.}
Всего функций от n переменных - $2^{n^n}$. Полиномов Жегалкина  $2^{n^n}$. Каждой функции соответсвует полином Жегалкина. У нас не может быть двух записей в виде полинома Жегалкина для одной и той же функции. Доказывается это от противного, если у нас есть два разных полинома Жегалкина  $Q_1$ и $Q_2$, которые задают одну и ту же функцию, то $Q_1\oplus Q_2 = 0$. Но раз $Q_1$ и $Q_2$ не совпадают, то $Q_1\oplus Q_2\neq0$. Тогда это биекция между двумя множествами. Q.E.D.
\end{enumerate}



\subsection{Классы Поста}
Классы Поста - это свойства булевых функций, такие, что если мы работаем над базисом, все функции которого принадлежат какому-то классу Поста, то и все функции, достижимые из этого базиса, также лежат в этом классе Поста.

\textbf{Классы Поста}:
\begin{enumerate}
    \item[1.] \textbf{Сохраняющие 0} - булевы функции, которые от входного набора состоящего только из 0 выдают 0.  $\{\vee, \wedge\}$ --- сохраняют ноль.
    
    Очевидно, любая функция над базисом из функций, сохраняющий 0 также будет сохранять 0.
    \item[2.] \textbf{Сохраняющие 1} - булевы функции, которые от входного набора состоящего только из 1 выдают 1.  $\{\vee, \wedge\}$ --- сохраняют один. 
    
    Также очевидно, что любая функция над базисом из функций, сохраняющий 1 также будет сохранять 1.
    \item[3.] \textbf{Линейность} - функция линейна, если ее представление в виде полинома Жегалкина не имеет в записи $\wedge$. 
    
    Доказательство линейности функции над базисом, состоящим из линейных функций также очевидно, если представить данную функцию в виде полинома Жегалкина, это упражнение я оставлю читателю.
    \item[4.] \textbf{Монотонность} - функция монотонна, если при увеличении любого аргумента результат не уменьшается, то есть $\forall$ 2 наборов аргументов $x_1, \ldots, x_n$ и $y_1, \ldots, y_n$, таких, что $x_i\leq y_i$, верно $f(x_1,x_2, \ldots, x_n) \leq f(y_1,y_2, \ldots, y_n)$ .
    
    Доказательство монотонности функции над базисом, состоящим из монотонных функций: Рассмотрим дерево операций для функии $f$, которая построена над базисом из монотонных функций. Тогда рассмотрим увеличение какого-то аргумента и докажем, что в этом случае значение $f$ не уменьшится. Предположим это не так, значит значение в корне уменьшилось. Но так как в корне у нас монотонная функция, то значение какого-то из его сыновей уменьшилось. Продолжим этот спуск и получим, что значение какого-то аргумента уменьшилось. Противоречие. Q.E.D. 
    \item[5.] \textbf{Самодвойственность} - функция самодвойственна, если $\forall$ набора аргументов $x_1, \ldots x_n$, верно  $F(x_1, \dots x_n)\neq F(\overline{x_1} \dots \overline{x_n})$. самодвойственности функции над базисом, состоящим из самодвойственных функций. Рассмотрим дерево операций для функии $f$, которая построена над базисом из самодвойственных функций. Тогда рассмотрим изменение всех аргументов и докажем, что в этом случае значение $f$ также изменится. Начнем рассматривать вершины снизу вверх. Так как каждая вершина - это самодвойственная функция, и все аргументы поменялись на противополжные, то значение и в этой вершине сменится. Тогда по итогу в корне значение также сменится на противоположное. Q.E.D. 
\end{enumerate}
\subsection{Критерий Поста}
$F$ --  полный базис $\Leftrightarrow$ нем есть не самодвойственная, не сохраняющая ноль, не сохраняющая один, не линейная, не монотонная.

\begin{enumerate}
\item[] \uline{Доказательство.}
\item[1)] Заметим, что необходимость этого утверждения была доказаны выше для каждого класса отдельно.

 \item[2)]Докажем, что если набор F не содержится полностью ни в одном из данных классов, то он является полным.

\begin{enumerate}
    \item  Рассмотрим функцию, не сохраняющую ноль --- $f_0$. Тогда $f_0(1)$ может принимать два значения:
     \begin{enumerate}
        \item[1.] $f_0(1)=1$, тогда $f_0(x,x,x,\ldots,x)=1$
        \item[2.] $f_0(1)=0$, тогда $f_0(x,x,x,\ldots,x)=\neg x$

.
    \end{enumerate}
    \item Рассмотрим функцию, не сохраняющую один — $f_1$. Тогда $f_1(0)$ может принимать два значения:
    \begin{enumerate}
        \item[1.]$f_1(0)=0$, тогда $f_1(x,x,x,\ldots,x)=0$
        \item[2.]$f_1(0)=1$, тогда $f_1(x,x,x,\ldots,x)=\neg x$
    \end{enumerate}
    Таким образом, возможны четыре варианта:
    \begin{enumerate}
        \item[a2b2.] Мы получили функцию $\neg$.
        Используем несамодвойственную функцию $f_s$. По определению, найдется такой вектор $x_0$, что $f_s(x_0)=f_s(\overline{x_0})$. Пусть $x_0=(x_{0_1},x_{0_2},\ldots,x_{0_k})$.
        
        Рассмотрим $f_s(x^{x_{0_1}},x^{x_{0_2}},\ldots,x^{x_{0_{k}}})$, где либо $x^{x_{0_i}}=x$, при $x_{0_i}=1$. Либо $x^{x_{0_i}}=\neg x$, при $x_{0_i}=0$. Нетрудно заметить, что $f_s(0)=f_s(1) \Rightarrow  f_s=const$. Таким образом мы получили одну из констант. Можем получить вторую, взяв отрицание от первой.

        \item[a2b1.] Мы получили $\neg$ и 0$\Rightarrow$ имеем константу, равную 1, поскольку $\neg 0=1$

        \item[a1b2.] Мы получили $\neg$ и 1$\Rightarrow$ имеем константу, равную 0, поскольку $\neg 1=0$

        \item[a1b1.] Мы получили 1 и 0. Рассмотрим немонотонную функцию $f_m$. Существуют такие $x_1,x_2,\ldots,x_n$, что $f_m(x_1,x_2,\ldots,x_{i-1},0,x_{i+1},\ldots,x_n)=1$, 
        $f_m(x_1,x_2,…,x_{i-1},1,x_{i+1},\ldots,x_n)=0$, зафиксируем все $x_1,x_2,…,x_n$, тогда $fm(x_1,x_2,…,x_{i-1},x,x_{i+1},\ldots,x_n)=\neg x$.  Такие   $x_1,x_2,\ldots,x_n$ являются x-ами при которых ломается монотонность

    \end{enumerate}
    В итоге  имеем три функции: $\neg, 0, 1$ в каждом из случаев. Покажем, что из них можно получить или или и.

    Используем нелинейную функцию $f_l$. Среди нелинейных членов $f_l$
 (ее представления в виде полинома Жегалкина), выберем тот, в котором минимальное количество элементов. 
 
 Все аргументы кроме двух в этом члене приравняем единице, оставшиеся два назовем $x_1$
 и $x_2$. 
 
 Все элементы, не входящие в данный член, примем равными нулю. Тогда эта функция будет представима в виде $g_l=x1x2[\oplus x1][\oplus x2][\oplus 1]$ 
, где в квадратных скобках указаны члены, которые могут и не присутствовать (остальные слагаемые будут равны нулю, поскольку в них есть как минимум один аргумент, не входящий в выбранный член, так как в выбранном члене минимальное число элементов).

Рассмотрим несколько вариантов:

\begin{enumerate}
    \item[1)] Присутствует член  $\oplus 1$.
    
    Возьмем отрицание от $g_l$ и член $\oplus 1$ исчезнет.
    \item[2)] Присутствуют три члена, без $\oplus 1:$
    
    $ g_l=x1x2\oplus x1 \oplus  x2$. Составив таблицу истинности для этой функции нетрудно заметить, что она эквивалентна функции или.
    \item[3)] Присутствуют два члена, без  $\oplus 1$.
    
    Построив две таблицы истинности для двух различных вариантов, заметим, что в обоих случаях функция истинна только в одной точке, следовательно, СДНФ функции $g_l$
 будет состоять только из одного члена. Если это так, то не составляет труда выразить или через не
 и $g_l$.
    \item[4)] Присутствует один член.
     Выразим $\wedge$ через $\neg$ и $g_l$ аналогично пункту 3.
\end{enumerate}

В итоге получим функцию $\neg$ ,а также функцию или либо и. Любую булеву функцию, не равную тождественному нулю, можно представить в форме СДНФ, то есть выразить в данном базисе.

Значит, полученные функции образуют полную систему, поскольку с их помощью можно выразить любую булеву функцию. Из этого следует, что F — полная система функций, что и требовалось доказать.

\end{enumerate}
\end{enumerate}

